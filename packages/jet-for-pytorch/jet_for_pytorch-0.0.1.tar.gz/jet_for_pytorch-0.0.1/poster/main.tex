\documentclass[a0paper]{tikzposter}
% Force a 16 by 9 ratio, 33.1 is same height as a0
\geometry{paperwidth=58.844in,paperheight=33.1in}

\usepackage{amsmath}
\usepackage{siunitx}
\usepackage{booktabs}
\usepackage{makecell} % table cells with multiple lines of text
\usepackage{multirow}
\usepackage{postertheme/vector_institute/vector_institute}

\definecolor{colorbrewer-green}{RGB}{27, 158, 119}
\colorlet{tab-green}{colorbrewer-green}
\definecolor{colorbrewer-orange}{RGB}{217, 95, 2}
\colorlet{tab-orange}{colorbrewer-orange}
\definecolor{colorbrewer-blue}{RGB}{117, 112, 179}
\colorlet{tab-blue}{colorbrewer-blue}

\input{../paper/preamble/goodfellow.tex}
\usetikzlibrary{calc,positioning}

% \useblockstyle{Envelope}
% \usebackgroundstyle{Empty}
\usetitlestyle{Filled}

\title{
  \fontsize{180}{198}\selectfont
  \bf Collapsing Taylor Mode Automatic Differentiation
}
\author{
  \fontsize{60}{66}\selectfont
  Felix Dangel*, Tim Siebert*, Marius Zeinhofer, Andrea Walther
}
\institute{
  \fontsize{50}{55}\selectfont
  Vector Institute (Canada),
  Humboldt University \& Zuse Institute Berlin (Germany),
  ETH Z\"urich (Switzerland)
}

% Make the most-used fonts bigger
\renewcommand{\normalsize}{\fontsize{40}{44}\selectfont}
\renewcommand{\large}{\fontsize{45}{49.5}\selectfont}
\renewcommand{\Large}{\fontsize{50}{55}\selectfont}
\renewcommand{\LARGE}{\fontsize{55}{60.5}\selectfont}


\begin{document}
% ==============================================================================
% HEADER & FOOTER

\backgroundgradient % Adds the background features
\maketitle
\headerlogo % Adds Vector logo to header
% \posterfooter{Poster footer with additional information} % Footer


\hspace{-35cm}
\begin{columns}
  \centering
  \begin{column}{1.85}
    \centering \ribbon{\centering\fontsize{120}{80}\selectfont\textcolor{white}{\bf
        \underline{TL;DR:}\quad We accelerate Taylor mode for practically relevant differential operators. \\[0.5ex]
        \hspace{-3.5ex}\underline{\phantom{;\!}How:}\quad By linearity-based graph rewrites that collapse Taylor coefficients.}}
  \end{column}
\end{columns}
\hspace{2cm}

\begin{columns}
  \column{0.598}
  \block{Background: What is Taylor Mode Autodiff?}{

    \begin{center}
      \textbf{Taylor mode generalizes forward mode to higher-order derivatives.}
    \end{center}

    \vspace{1.5ex}
    \begin{center}
      \includegraphics[width=0.7\linewidth]{figures/taylor}
    \end{center}

    \vspace{1ex}

    Fa\`a di-Bruno's formula describes the propagation of Taylor coefficients (jets). Choosing different Taylor coefficients computes different derivatives.

    \vspace{1.5ex}
    \begin{center}
      \textbf{Only JAX implements Taylor mode. We implement it for PyTorch.}
      \vspace{0.5ex}

      \texttt{pip install jet-for-pytorch}
    \end{center}

    \vspace{0.85ex}

    \begin{center}
      \textbf{For PDE operators, we typically need to run Taylor mode along multiple directions, then sum the resulting derivatives.}
    \end{center}

    \vspace{0.85ex}

    \begin{center}
      \begin{tikzpicture}
        \node [inner xsep=80pt, inner ysep=30pt, draw=tab-orange, rounded corners=15pt, line width=10pt] (graph-before) {\includegraphics[height=11.5cm]{figures/sin_2jet_0}};
        \coordinate (separator) at ($(graph-before.north east)!0.46!(graph-before.south east)$);
        \draw[tab-orange, fill=tab-orange, line width=5pt, rounded corners=15pt] (graph-before.south west) rectangle (separator);
        % Re-draw
        \node [inner xsep=80pt, inner ysep=30pt, draw=tab-orange, rounded corners=15pt, line width=10pt] {\includegraphics[height=11.5cm]{figures/sin_2jet_0}};
        \node [anchor=north east, align=center] at (separator) {\textbf{Propagate then sum}};
      \end{tikzpicture}
    \end{center}
  }

  \column{0.598}

  \block{}{
    \begin{minipage}[c]{0.5\linewidth}
      \centering
      \LARGE \bfseries
      Interested in similar topics?

      Felix is hiring for 2026!

      % (Concordia University \& Mila, Montreal)
    \end{minipage}
    \begin{minipage}[c]{0.49\linewidth}
      \includegraphics[height=3cm]{logos/logo-concordia.pdf}
      % [trim={left bottom right top},clip]
      \includegraphics[trim={6cm 8cm 6cm 8cm},clip, height=3cm]{logos/logo-mila.png}

    \end{minipage}
  }

  \block{Main Idea: Use Linearity to Collapse the Propagation}{

    \begin{center}
      \textbf{Linearity is nice because order does not matter (associativity).}
    \end{center}
    \vspace{2.5ex}
    \begin{align*}
      \begin{array}{c}
        \text{Given a matrix $\mW$}
        \\
        \text{and vectors $\{\vx_n\}$,}
        \\
        \text{what is better?}
      \end{array}
      \qquad
      \begin{array}{c c c}
        \displaystyle
        \color{tab-orange}
        \sum_n \left( \mW \vx_n \right)
        & \qquad\text{or}\qquad
        & \color{tab-green}
          \displaystyle \mW \Big( \sum_n \vx_n \Big)
        \\
        \color{tab-orange}
        \text{\normalsize \textbf{(act then sum)}}
        &
        &
          \color{tab-green}
          \text{\normalsize \textbf{(sum then act)}}
      \end{array}
    \end{align*}

    \vspace{2.9ex}

    \textbf{Our insight:} Fa\`a di-Bruno's formula exhibits linearity in the sum of highest Taylor components.
    We can choose the better order and propagate the summed Taylor coefficient.
    \textbf{We call this \textcolor{tab-green}{`collapsing' Taylor mode}.}

    \vspace{2.5ex}

    \textbf{Collapsing reduces FLOPs, memory, and can be automated (e.g.\,in \texttt{jit}).}

    \vspace{2ex}

    % Example: Collapsing multiple 2-jets of the sine function.
    \begin{center}
      % \resizebox{!}{12cm}{
      % {\tikzset{%
      % every matrix/.append style={column sep=2ex, row sep=2.5ex},%
      % every node/.append style={scale=0.75},%
      % every path/.append style={line width=2.5pt},%
      % Stealth/.append style={scale=2},%
      % }
      \begin{tikzpicture}
        \node [inner xsep=80pt, inner ysep=30pt, draw=tab-green, rounded corners=15pt, line width=10pt] (graph-after) {\includegraphics[height=11.5cm]{figures/sin_2jet_3}};
        \coordinate (separator) at ($(graph-after.north east)!0.46!(graph-after.south east)$);
        \draw [tab-green, line width=5pt, rounded corners=15pt, fill=tab-green] (graph-after.south west) rectangle (separator);
        % Re-draw
        \node [inner xsep=80pt, inner ysep=30pt, draw=tab-green, rounded corners=15pt, line width=10pt] {\includegraphics[height=11.5cm]{figures/sin_2jet_3}};
        \node [anchor=north east, align=center] at (separator) {\textbf{Sum then propagate}};
      \end{tikzpicture}
      % }
      % }
    \end{center}
  }


  \column{0.598}
  \block{Scope \& Performance}{
    \begin{minipage}[t]{0.45\linewidth}
      \begin{center}
        \textbf{Collapsed Taylor mode contains and improves existing methods.}
      \end{center}

      \vspace{1.5ex}

      \begin{itemize}
      \item \textcolor{tab-orange}{\textbf{Laplacian}}
        \\
        {\color{tab-green} $\Leftrightarrow$ \textbf{forward Laplacian}}
        \\[0.1ex]

      \item \textcolor{tab-orange}{\textbf{Stochastic Taylor mode}}
        \\
        {\color{tab-green}$\to$ \textbf{collapses samples}}
        \\[0.1ex]

      \item \textcolor{tab-orange}{\textbf{General linear PDE operators}}
        \\
        {\color{tab-green}$\to$ \textbf{collapses interpolation}}
      \end{itemize}
      \vspace{1.5ex}
      \begin{center}
        \textbf{Collapsing improves vanilla Taylor mode and outperforms nesting  first-order autodiff.}
      \end{center}

    \end{minipage}
    \hfill
    \begin{minipage}[t]{0.495\linewidth}
      \centering
      \vspace*{-1ex}
      \includegraphics[width=0.9\linewidth]{figures/overview.pdf}
    \end{minipage}
    \begin{center}
    \end{center}

    \vspace{1ex}

    \begin{center}
      % paths where the performances are stored
\def\datapathLaplacianExact{../jet/exp/exp01_benchmark_laplacian/performance/architecture_tanh_mlp_768_768_512_512_1_device_cuda_dim_50_name_laplacian_vary_batch_size}
\def\datapathLaplacianStochastic{../jet/exp/exp01_benchmark_laplacian/performance/architecture_tanh_mlp_768_768_512_512_1_batch_size_2048_device_cuda_dim_50_distribution_normal_name_laplacian_vary_num_samples}
\def\datapathWeightedLaplacianExact{../jet/exp/exp01_benchmark_laplacian/performance/architecture_tanh_mlp_768_768_512_512_1_device_cuda_dim_50_name_weighted_laplacian_vary_batch_size_rank_ratio_1_0}
\def\datapathWeightedLaplacianStochastic{../jet/exp/exp01_benchmark_laplacian/performance/architecture_tanh_mlp_768_768_512_512_1_batch_size_2048_device_cuda_dim_50_distribution_normal_name_weighted_laplacian_vary_num_samples_rank_ratio_1_0}
\def\datapathBilaplacianExact{../jet/exp/exp01_benchmark_laplacian/performance/architecture_tanh_mlp_768_768_512_512_1_device_cuda_dim_5_name_bilaplacian_vary_batch_size}
\def\datapathBilaplacianStochastic{../jet/exp/exp01_benchmark_laplacian/performance/architecture_tanh_mlp_768_768_512_512_1_batch_size_256_device_cuda_dim_5_distribution_normal_name_bilaplacian_vary_num_samples}
% configuration options for the \num command
\sisetup{%
  % scientific-notation=true,%
  round-mode=figures,%
  round-precision=2,%
  detect-weight, % for bolding to work
  tight-spacing=true, % less space around \cdot
}
\resizebox{0.84\linewidth}{!}{
  \begin{tabular}{ccc|cccc}
    \toprule
    \textbf{Mode}
    & \makecell{\textbf{Per-datum/sample}}
    & \textbf{Implementation}
    & \textbf{Laplacian}
    & \makecell{\textbf{Weighted} \textbf{Laplacian}}
    & \textbf{Biharmonic}
    \\
    \midrule
    \multirow{6}{*}{\textbf{Exact}}
    & \multirow{3}{*}{Time [ms]}
    & \textcolor{tab-blue}{Nested 1\textsuperscript{st}-order}
    & \input{\datapathLaplacianExact/hessian_trace_best.txt}
    & \input{\datapathWeightedLaplacianExact/hessian_trace_best.txt}
    & \input{\datapathBilaplacianExact/hessian_trace_best.txt}
    \\
    &
    & \textcolor{tab-orange}{Standard Taylor}
    & \input{\datapathLaplacianExact/jet_naive_best.txt}
    & \input{\datapathWeightedLaplacianExact/jet_naive_best.txt}
    & \input{\datapathBilaplacianExact/jet_naive_best.txt}
    \\
    &
    & \textcolor{tab-green}{\textbf{Collapsed (ours)}}
    & \textbf{\input{\datapathLaplacianExact/jet_simplified_best.txt}}
    & \textbf{\input{\datapathWeightedLaplacianExact/jet_simplified_best.txt}}
    & \textbf{\input{\datapathBilaplacianExact/jet_simplified_best.txt}}
    \\ \cmidrule{2-6}
    & \multirow{3}{*}{\makecell{Mem.\,[MiB]}}
    & \textcolor{tab-blue}{Nested 1\textsuperscript{st}-order}
    & \input{\datapathLaplacianExact/hessian_trace_peakmem.txt}
    & \input{\datapathWeightedLaplacianExact/hessian_trace_peakmem.txt}
    & \input{\datapathBilaplacianExact/hessian_trace_peakmem.txt}
    \\
    &
    & \textcolor{tab-orange}{Standard Taylor}
    & \input{\datapathLaplacianExact/jet_naive_peakmem.txt}
    & \input{\datapathWeightedLaplacianExact/jet_naive_peakmem.txt}
    & \input{\datapathBilaplacianExact/jet_naive_peakmem.txt}
    \\
    &
    & \textcolor{tab-green}{\textbf{Collapsed (ours)}}
    & \textbf{\input{\datapathLaplacianExact/jet_simplified_peakmem.txt}}
    & \textbf{\input{\datapathWeightedLaplacianExact/jet_simplified_peakmem.txt}}
    & \textbf{\input{\datapathBilaplacianExact/jet_simplified_peakmem.txt}}
    \\
    \midrule
    \multirow{6}{*}{\textbf{Stochastic}\,\,}
    & \multirow{3}{*}{Time [ms]}
    & \textcolor{tab-blue}{Nested 1\textsuperscript{st}-order}
    & \input{\datapathLaplacianStochastic/hessian_trace_best.txt}
    & \input{\datapathWeightedLaplacianStochastic/hessian_trace_best.txt}
    & \input{\datapathBilaplacianStochastic/hessian_trace_best.txt}
    \\
    &
    & \textcolor{tab-orange}{Standard Taylor}
    & \input{\datapathLaplacianStochastic/jet_naive_best.txt}
    & \input{\datapathWeightedLaplacianStochastic/jet_naive_best.txt}
    & \input{\datapathBilaplacianStochastic/jet_naive_best.txt}
    \\
    &
    & \textcolor{tab-green}{\textbf{Collapsed (ours)}}
    & \textbf{\input{\datapathLaplacianStochastic/jet_simplified_best.txt}}
    & \textbf{\input{\datapathWeightedLaplacianStochastic/jet_simplified_best.txt}}
    & \textbf{\input{\datapathBilaplacianStochastic/jet_simplified_best.txt}}
    \\ \cmidrule{2-6}
    & \multirow{3}{*}{\makecell{Mem.\,[MiB]}}
    & \textcolor{tab-blue}{Nested 1\textsuperscript{st}-order}
    & \input{\datapathLaplacianStochastic/hessian_trace_peakmem.txt}
    & \input{\datapathWeightedLaplacianStochastic/hessian_trace_peakmem.txt}
    & \input{\datapathBilaplacianStochastic/hessian_trace_peakmem.txt}
    \\
    &
    & \textcolor{tab-orange}{Standard Taylor}
    & \input{\datapathLaplacianStochastic/jet_naive_peakmem.txt}
    & \input{\datapathWeightedLaplacianStochastic/jet_naive_peakmem.txt}
    & \input{\datapathBilaplacianStochastic/jet_naive_peakmem.txt}
    \\
    &
    & \textcolor{tab-green}{\textbf{Collapsed (ours)}}
    & \textbf{\input{\datapathLaplacianStochastic/jet_simplified_peakmem.txt}}
    & \textbf{\input{\datapathWeightedLaplacianStochastic/jet_simplified_peakmem.txt}}
    & \textbf{\input{\datapathBilaplacianStochastic/jet_simplified_peakmem.txt}}
    \\
    \bottomrule
  \end{tabular}
}
%%% Local Variables:
%%% mode: LaTeX
%%% TeX-master: "../main"
%%% End:

    \end{center}
  }
\end{columns}

\end{document}
%%% Local Variables:
%%% mode: LaTeX
%%% TeX-master: t
%%% End:
