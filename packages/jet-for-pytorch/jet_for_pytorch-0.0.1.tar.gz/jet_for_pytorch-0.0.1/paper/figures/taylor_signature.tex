\begin{figure}[!t]
  \centering
  \begin{minipage}[t]{0.63\linewidth}
    \centering
    \vspace{0pt}
   \begin{tikzpicture}
    \tikzset{box/.style={rectangle, rounded corners, draw=black, inner sep=3pt, very thick}}
    \node[align=center, box] (topleft) {Extent input to \\ smooth path
      $\vx(t)$};

    \node[align=center, right=2.7cm of topleft, box] (topright) {Path in output \\ space $\vf(\vx(t))$};
    \draw [-Latex] (topleft.east) to node [midway, above] {$\vf$} (topright.west);

    \node[align=center, below=0.9cm of topright, box, draw=tab-orange, fill=tab-orange!25!white] (bottomright) {%Taylor polynomial of degree $K$
      % \\
      $K$-jet \; $\sum_{k=0}^K \frac{t^k}{k!} \vf_k$
      \\
      as $(\vf_0, \dots, \vf_K)$
    };
    \draw [-Latex] (topright.south) to node [midway, right] {$J^K$}
    (bottomright.north);

    \node[align=center, below=0.9cm of topleft, box, draw=tab-orange, fill=tab-orange!25!white] (bottomleft) {%Taylor polynomial of degree $K$
        % \\
        $K$-jet \; $\sum_{k=0}^K \frac{t^k}{k!} \vx_k$
        \\
        as $(\vx_0, \dots, \vx_K)$
      };
    \draw [-Latex] (topleft.south) to node [midway, left] {$J^K$} (bottomleft.north);

    \draw [-Latex, \colorTM, align=center, very thick] (bottomleft.east) to node [midway, above, tab-orange] {\color{\colorTM}Taylor mode} (bottomright.west);
  \end{tikzpicture}
  \end{minipage}
  \hfill
  \begin{minipage}[t]{0.35\linewidth}
    \caption{\textbf{Taylor mode propagates Taylor coefficients of a path in input space.}
    This results in the function-transformed path's Taylor coefficients.
    The Taylor expansion of degree $K$ is called a $K$-jet; hence Taylor mode propagates the input $K$-jet to the output $K$-jet.}
  \label{fig:utp}
  \end{minipage}
\end{figure}

%%% Local Variables:
%%% mode: LaTeX
%%% TeX-master: "../main"
%%% End:
