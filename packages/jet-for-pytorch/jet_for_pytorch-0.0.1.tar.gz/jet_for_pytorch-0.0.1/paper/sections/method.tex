We now describe how to collapse the Taylor mode AD computation of popular linear PDE operators and their stochastic approximations proposed in \cite{shi2024stochastic}, and provide a general recipe for computing and collapsing general linear differential operators by interpolation, using earlier work from \citet{griewank_evaluating_1999}.
At its core, our procedure
uses the linearity of the highest Taylor coefficient's propagation rule.
It allows to collapse coefficients along multiple directions and directly \textcolor{tab-green}{\bfseries propagate their sum}, rather than \textcolor{tab-orange}{\bfseries propagating then summing},
yielding substantial reductions in computational cost.

\subsection{Exploiting Linearity to Collapse Taylor Mode AD}

To derive our proposed method, we start with a sum of $K$th-order directional derivatives of the function $\vf$ along $R$ directions $\{\vv_r\}_{r=1}^R$, which is a common building block for all our PDE operators:
\begin{align}\label{eq:sum-k-directional}
  {\color{tab-orange}
  \sum_{r=1}^R
  }
  \left<
  \partial^{K} \vf(\vx_0),
  \vv_r^{\otimes K}
  \right> \, \in \sR^C
\end{align}
(\eg, the exact Laplacian uses $K = 2$, $R = \dim(\vx_0) = D$, and the unit vectors $\vv_r = \ve_r\in \R^D$ as directions; see \cref{sec:second-order-operators} below).
Instead of nesting $K$ calls to 1\textsuperscript{st}-order AD, we can use $K$-jets to calculate each summand of \cref{eq:sum-k-directional} with Taylor mode.
In total, we need $R$ $K$-jets, and have to set the $r$-th jet's coefficients to $\vx_{0, r} = \vx_0$, $\vx_{1, r} = \vv_r$ and $\vx_{2, r} = \ldots = \vx_{K, r} = \vzero$ (\cref{eq:sum-taylor-mode-naive} applies this to \cref{eq:taylor-mode-composition}).

Standard Taylor mode propagates $1 + KR$ vectors through every node of the computational graph (the $0$th component is shared across all jets, see \cref{fig:visual-abstract}).
This gives the output jets $\smash{\{\{\vf_{k,r}\}_{k=1}^K\}_{r=1}^{R}}$, from which we only select the highest-degree coefficients $\smash{\{\vf_{K,r}\}_{r=1}^R}$, then sum them to obtain \cref{eq:sum-k-directional}.

The approach we propose here exploits that the $K$-th derivative of $\vg \circ \vh$ is $\partial \vg$ times the $K$-th derivative of $\vh$ plus other lower-order terms in $\vh$. Therefore, $\vg \circ \vh$ is linear in the $K$-th derivative of $\vh$. Mathematically speaking, there is a special element in the set of integer partitions $\partitioning(K)$, namely the trivial partition $\tilde{\sigma}= \{K\}$, which contributes the term $ \nu(\tilde{\sigma}) \left< \partial \vg(\vh_0), \vh_{K,r} \right>$ to \cref{eq:faa-di-bruno}.
This is the only term that uses the input jet's highest coefficient $\vh_{K,r}$, and its dependency is \emph{linear}.
Separating it in the highest coefficient's forward propagation, we get (using $\nu(\tilde{\sigma}) = 1$)
\begin{align}
  %\textstyle % Comment out this line if we have enough space
  \nonumber
  \sum_{r=1}^R\vf_{K, r}
  =
  \sum_{r=1}^R\vg_{K,r}
  &=
  {
  \color{tab-orange}
  \sum_{r=1}^R
  }
  \sum_{
    \sigma \in \partitioning(K) \setminus \{\tilde{\sigma}\}
  }\!\!\!\!
  \nu(\sigma) \left<\!
    \partial^{|\sigma|} \vg(\vh_0),
    \tensorprod{s \in \sigma} \vh_{s, r}\!
  \right>
  \!+\!
  {
  \color{tab-orange}
  \sum_{r=1}^R
  }
  \left<
    \partial \vg(\vh_0),
    \vh_{K, r}
  \right>
  \shortintertext{and since the propagation rule is linear \wrt $\vh_{K,r}$, we can pull the summation inside:}
  {
  \color{tab-green}
  \sum_{r=1}^R
  }
  \vg_{K,r}
  &=
  {\color{tab-orange}
  \sum_{r=1}^R
  }
  \sum_{
    \sigma \in \partitioning(K) \setminus \{\tilde{\sigma}\}
  }\!\!\!\!
  \nu(\sigma) \left<\!
    \partial^{|\sigma|} \vg(\vh_0),
    \tensorprod{s \in \sigma} \vh_{s, r}\!
  \right>
  \!+\!
  \left<\!
    \partial \vg(\vh_0),
    {
    \color{tab-green}\sum_{r=1}^R}
    \vh_{K, r}
  \right>
  \,.\label{eq:faa-di-bruno-expanded}
\end{align}
This is the key insight of our work:
\emph{The summed highest-degree output coefficients depend on the summed highest-degree input coefficients} (as well as all lower-degree coefficients).
The reason is \emph{linearity} in Fa\`a di Bruno's formula.
Hence, to compute the sum $\sum_r \vg_{K, r}$ we can directly propagate the sum $\sum_r \vh_{K,r}$, collapsing coefficients over all directions.
We call this \emph{collapsed Taylor mode AD}.

Collapsed Taylor mode propagates only $1 + (K - 1)R + 1$ vectors through every node in the computational graph (see \cref{fig:visual-abstract} and \cref{eq:sum-taylor-mode-efficient} which applies this to \cref{eq:faa-di-bruno}).
These savings of $R-1$ coefficients are significant improvements over standard Taylor mode, as we show below.
In the following, we discuss how to collapse the Taylor mode computation of various PDE operators.

\subsection{Linear Second-order Operators}\label{sec:second-order-operators}

\paragraph{Laplacian.} The Laplace operator plays a central role in Physics and engineering, including electrostatics, fluid dynamics, heat conduction, and quantum mechanics \cite{foulkes2001quantum, pfau2020ab}.
It contains the Hessian trace of each element of a function, \ie, for $\vf: \sR^D \to \sR^C$, it is
\begin{subequations}
  \begin{align}\label{eq:laplacian}
    \underbrace{
    \Delta \vf(\vx_0)
    }_{\in \sR^C}
    :=
    \left<
    \partial^2 \vf(\vx_0), \mI_D
    \right>
    \quad
    \begin{cases}
      \displaystyle
      = \sum_{d=1}^D \left<
      \partial^2 \vf(\vx_0),
      \ve_d^{\otimes 2}
      \right>
      &\text{(exact)}
      \\
      \displaystyle
      \overset{\text{\cite{shi2024stochastic}}}{\approx}
      \frac{1}{S} \sum_{s=1}^S
      \left<
      \partial^2 \vf(\vx_0),
      \vv_s^{\otimes 2}
      \right>
      &\text{(stochastic)}
    \end{cases}
  \end{align}
  with the $d$-th standard basis vector $\ve_d$ used for exact computation, and $S$ random vectors $\vv_s$ drawn \iid from a distribution with unit variance (\eg Rademacher or standard Gaussian) for stochastic estimation.
  By pattern-matching \cref{eq:laplacian} with \cref{eq:sum-k-directional} we conclude that $K=2$, and the following choices for computing the Laplacian with standard Taylor mode:
  \begin{align}
    \begin{array}{rlrlrlll}
      \{(\vx_{0,d} & \!\!\!\!= \vx_0, &\vx_{1, d} & \!\!\!\!= \ve_d, & \vx_{2,d} &\!\!\!\!= \vzero)\}_{d=1}^D
                                                                   \quad
      &{\color{tab-orange} 1 + D + D\, \text{vectors}}
        \quad
      &\text{(exact)}
      \\[1ex]
      \{(\vx_{0,s} & \!\!\!\!= \vx_0, &\vx_{1, s} & \!\!\!\!= \vv_s, &\vx_{2,s} & \!\!\!\!= \vzero)\}_{s=1}^S
                                                                  \quad
      &{\color{tab-orange} 1 + S + S\, \text{vectors}}
        \quad
      &\text{(stochastic)}
    \end{array}.
  \end{align}
\end{subequations}
Collapsing standard Taylor mode yields {\color{tab-green}$1 + D + 1$} (exact) and {\color{tab-green}{$1 + S + 1$}} (stochastic) vectors.
In fact, the collapsed Taylor mode for the exact Laplacian is the forward Laplacian from \citet{li2023forward} (see \cref{eq:laplacian-efficient} for detailed presentation of the forward propagation).
Note how we can seamlessly also collapse the stochastic approximation over the sampled directions, which is currently not done.

\paragraph{Weighted Laplacian.}
A natural generalization of the Laplacian involves contracting with a positive semi-definite matrix $\mD = \msigma \msigma^\top \in \mathbb R^{D\times D}$ rather than the identity. $\mD$ may represent the diffusion tensor in Kolmogorov-type PDEs like the Fokker-Planck equation \cite{hu2024hutchinson}, and $\msigma$ can depend on $\vx_0$ \cite{fa2011solution}.
The weighted Laplacian contains the weighted Hessian's trace $\Tr(\msigma \msigma^{\top} \partial^2 [\vf]_c)$ for each output element $c$ of $\vf$.
If $\rank(\mD) = R$ and therefore $\msigma = (\vs_1, \dots, \vs_R) \in \sR^{D \times R}$, it is
\begin{subequations}
  \begin{align}\label{eq:weighted-laplacian}
    \underbrace{
    \Delta_\mD \vf(\vx_0)
    }_{\in \sR^C}
    :=
    \left<
    \partial^2 \vf(\vx_0), \mD
    \right>
    \quad
    \begin{cases}
      \displaystyle
      = \sum_{r=1}^R
      \left< \partial^2 \vf(\vx_0), \vs_r^{\otimes 2} \right>
      & \text{(exact)}
      \\
      \displaystyle
      \overset{\text{\cite{hu2024hutchinson}}}{\approx}
      \frac{1}{S}
      \sum_{s=1}^S
      \left< \partial^2 \vf(\vx_0), (\msigma \vv_s)^{\otimes 2} \right>
      & \text{(stochastic)}
    \end{cases}
    .
  \end{align}
  Computing it requires evaluating the following 2-jets with standard Taylor mode:
  \begin{align}
    \begin{array}{rlrlrlll}
      \{(\vx_{0,r} & \!\!\!\!= \vx_0, &\vx_{1, r} & \!\!\!\!= \vs_r, & \vx_{2,r} &\!\!\!\!= \vzero)\}_{r=1}^R
                                                                   \quad
      &{\color{tab-orange} 1 + R + R\, \text{vectors}}
        \quad
      &\text{(exact)}
      \\[1ex]
      \{(\vx_{0,s} &\!\!\!\!= \vx_0, &\vx_{1, s} & \!\!\!\!= \msigma \vv_s, &\vx_{2,s} & \!\!\!\!= \vzero)\}_{s=1}^S
                                                                          \quad
      &{\color{tab-orange} 1 + S + S\, \text{vectors}}
        \quad
      &\text{(stochastic)}
    \end{array}.
  \end{align}
\end{subequations}
Our collapsed Taylor mode uses {\color{tab-green}$1 + R + 1$} (exact) and {\color{tab-green}{$1 + S + 1$}} (stochastic) vectors.
This yields the modified forward Laplacian from \citet{li2024dof}; collapsing the stochastic variant speeds up the Hutchinson trace estimator from \citet{hu2024hutchinson}.
For indefinite $\mD$, we can simply apply this scheme to the positive and negative eigen-spaces (however, such weightings are not used in practise).

\subsection{Collapsed Taylor Mode for Arbitrary Mixed Partial Derivatives}
So far, we discussed operators that result from contracting the second-order derivative tensor with a coefficient matrix ($\mI$ or $\mD$) that can conveniently be written as sum of vector outer products.
For orders higher than two, the coefficient tensor can in general \emph{not} easily be decomposed as such.
Hence, we extend our framework to also cover differential operators containing mixed-partial derivatives by evaluating a suitable family of jets using the interpolation result of \citet{griewank_evaluating_1999}.
As illustrative example, we will use the biharmonic operator with a 4-dimensional coefficient tensor:
\begin{align}
  \label{eq:biharm}
  \underbrace{
  \Delta^2 \vf(\vx_0)
  }_{\in \sR^C}
  % \coloneqq
  % \Delta(\Delta \vf(\vx_0))
  \quad
  \begin{cases}
    \displaystyle
    =
    \sum_{d_1=1}^D \sum_{d_2=1}^D
    \left<
    \partial^4 \vf(\vx_0),
    \ve_{d_1}^{\otimes 2} \otimes \ve_{d_2}^{\otimes 2}
    \right>
    & \text{(exact)}
    \\
    \displaystyle
    \overset{\text{\cite{shi2024stochastic}}}{\approx}
    \frac{1}{3S}
    \sum_{s=1}^S
    \left< \partial^4 \vf(\vx_0), \vv_s^{\otimes 4} \right>
    & \text{(stochastic)}
  \end{cases}\,.
\end{align}
We can directly collapse the stochastic version: draw $S$ standard normal vectors $\vv_1, \dots, \vv_S$ and propagate the coefficients $\{(\vx_{0,s}= \vx_0, \vx_{1,s} = \vv_s, \vx_{2,s} = \vx_{3,s} = \vx_{4,s} = \vzero)\}_{s=1}^S$.
With standard Taylor mode, this uses ${\color{tab-orange} 1 + 4S}$ vectors; collapsed Taylor mode uses ${\color{tab-green} 1 + 3S + 1}$ vectors.
For the exact biharmonic operator, however, we need to develop an approach to compute mixed partials.

\paragraph{General approach.}
Assume we want to compute a linear differential operator of degree $K$.
We can do so by contracting the $K$-th order derivative tensor $\partial^K \vf(\vx_0)$ with a coefficient tensor $\tC \in (\sR^D)^{\otimes K}$.
We can always express this tensor in a tensor product basis, such that
\begin{align}\label{eq:sums-k-directional}
  \left<
  \partial^K \vf(\vx_0), \tC
  \right>
  =
  \sum_{d_1=1}^{D_1} \dots \sum_{d_I=1}^{D_I}\left<
  \partial^{K} \vf(\vx_0),
  \vv_{d_1}^{\otimes i_1}
  \otimes \ldots \otimes
  \vv_{d_I}^{\otimes i_I}
  \right>
  \,
  \in \sR^C\,,
\end{align}
where the multi-index entries $\vi = (i_1, \dots, i_I)$ sum to $K$ and $D_j \leq D$.
For the exact biharmonic operator (\cref{eq:biharm}), we identify $K = 4, I = 2, \vi = (2, 2), D_1 = D_2 = D, \vv_{d_1} = \ve_{d_1}$, and $\vv_{d_2} = \ve_{d_2}$.
From the Fa\'a di Bruno formula, we know that we can only compute terms of the form $\langle \partial^{K} \vf(\vx_0), \vv^{\otimes K}\rangle$ with a $K$-jet.
The challenge in \cref{eq:sums-k-directional} is that it includes terms where \emph{not} all directions coincide (\eg, for the biharmonic we have $I=2$ different directions).

Fortunately, \citet{griewank_evaluating_1999} derived an approach to reconstruct such mixed-direction terms by linearly combining a \emph{family} of $K$-jets that is determined by all vectors $\vj \in \sN^I$ whose entries sum to $K$, see \cref{fig:ttc_biharm_coeffs} for an illustration for the biharmonic (5 members).
The $K$-jets along these directions are then combined with coefficients $\gamma_{\vi, \vj} \in \sR$, whose definition we provide in \cref{sec:appendix_ttc}.
In summary, we get
\begin{equation}
  \label{eq:ttc_general}
  \!\!\!\!\!
  \left<
    \partial^{K} \vf(\vx_0),
    \vv_{d_1}^{\otimes i_1}
    \otimes \ldots \otimes
    \vv_{d_I}^{\otimes i_I}
  \right>
  = \sum_{\vj \in \sN^I, \lVert \vj \rVert_1 = K}
  \frac{\gamma_{\vi, \vj}}{K!}
  \left<
    \partial^{K}\vf(\vx_0),
    \left(\sum_{i=1}^I \vv_{d_i} [\vj]_i\right)^{\otimes K}
  \right>\,.
\end{equation}
This construction allows us to rewrite \cref{eq:sums-k-directional} as
\begin{align*}
  %\textstyle % Comment out this line if we have enough space
  \sum_{d_1=1}^{D_1} \dots \sum_{d_I=1}^{D_I}
  \sum_{\vj \in \mathbb{N}^I, \lVert \vj \rVert_1 = K}
  \frac{\gamma_{\vi, \vj}}{K!}
  \left<
  \partial^{K}\vf(\vx_0),
  \left(
  \sum_{i = 1}^I \vv_{d_i} [\vj]_i
  \right)^{\otimes K}
  \right>\,,
\end{align*}
and---since the coefficients $\gamma_{\vi,\vj}$ only depend on the problem structure ($K$, $I$ and $\vi$) and \emph{not} on the function $\vf$ and the directions $\vv_{d_i}$ \cite{griewank_evaluating_1999}---we can pull out the inner sum to obtain the final expression
\begin{equation}\label{eq:ttc-general}
  \sum_{\vj \in \mathbb{N}^I, \lVert \vj \rVert_1 = K}
  \frac{\gamma_{\vi, \vj}}{K!}
  {\color{tab-orange}
  \sum_{d_1=1}^{D_1} \dots \sum_{d_I=1}^{D_I}
  }
  \left<
    \partial^{K}\vf(\vx_0),
    \left(
      \sum_{i = 1}^I \vv_{d_i} [\vj]_i
    \right)^{\otimes K}
  \right>\,.
\end{equation}

We can evaluate \cref{eq:ttc-general} with standard Taylor mode: For each $\vj$, compute $\smash{\prod_{i=1}^{I}}D_i$ $K$-jets with coefficients $\vx_0, \vx_1 = \sum_i \vv_{d_i}[\vj]_i, \vx_2= \ldots =\vx_K = \vzero$.
The sums from the tensor basis expansion can be collapsed with our proposed optimization, removing $\smash{\prod_{i=1}^{I}}D_i$ vectors from the propagation for each $\vj$.
After repeating for each member $\vj$ of the interpolation family, we form the linear combination using the $\gamma_{{\vi, \vj}}$s, which yields the desired differential operator.
We can often exploit symmetries in the $\gamma_{\vi, \vj}$s and basis vectors to further reduce the number of $K$-jets (see \cref{sec:appendix-biharmonic-details} for a complete example).

\paragraph{Applied to the biharmonic operator.}
Let us now illustrate the key steps of applying \cref{eq:ttc-general} to the exact biharmonic operator \cref{eq:biharm} (full procedure in \cref{sec:appendix-biharmonic-details}).
\Cref{fig:ttc_biharm_coeffs} illustrates the 5 multi-indices $\vj$ characterizing the $4$-jets we need to interpolate $\langle \partial^4 \vf(\vx_0), \ve_{d_1}^{\otimes 2} \otimes \ve_{d_2}^{\otimes 2} \rangle$, and their coefficients $\gamma_{\vi, \vj}$.
Their definition, see \cref{eq:ttc_coeff}, shows the equality of $\gamma_{\vi, \vj}$ for $\vj = (4,0)$ and $\vj = (0, 4)$, as well as $\vj = (3, 1)$ and $\vj = (1, 3)$.
Exploiting those symmetries reduces the number of interpolation terms from 5 to 3 (\cref{eq:ttc_for_biharm_2}), corresponding to $D + D^2 + D^2$ $4$-jets.
Removing doubly-computed terms brings down the number of $4$-jets to $D + D(D-1) + \nicefrac{1}{2}D(D-1)$ (\cref{eq:ttc_for_biharm_final}).
Translated to vectors, standard Taylor mode propagates $1 + 4D + 4D(D-1) + \nicefrac{4}{2}D(D-1) = {\color{tab-orange} 6D^2 - 2D + 1}$ vectors.
After collapsing, we get $1 + 3D + 1 + 3 D(D-1) + 1 + \nicefrac{3}{2} D(D-1)  + 1 = {\color{tab-green}\nicefrac{9}{2}D^2 - \nicefrac{3}{2} D + 4}$ vectors.
\begin{wrapfigure}[17]{r}{0.415\textwidth}
  \centering
  \vspace*{-0.5ex}
  \begin{tikzpicture}
  \tikzset{transparentfill/.style={fill=white, fill opacity=0.7, text opacity=1.0}}
  \begin{axis}[
    axis lines=middle,
    axis line style={-Stealth, thick},
    xmin=0, xmax=5.25,
    ymin=0, ymax=5.25,
    xtick={0,1,2,3,4},
    ytick={0,1,2,3,4},
    grid=major,
    %xlabel={Along $\ve_{d_1}$},
    %ylabel={Along $\ve_{d_2}$},
    enlargelimits=false,
    tick style={black},
    tick align=outside,
    tick label style={font=\small},
    width=0.4\textwidth,
    height=0.4\textwidth,
    xlabel style={at={(axis description cs:0.5,-0.15)},anchor=north, font=\small},
    ylabel style={at={(axis description cs:-0.15,0.5)},anchor=south, rotate=90, font=\small},
    clip=false, %  no clipping if something is outside the axis
    ]

    % Loop to draw circles at (x, y) such that x + y = 4
    \foreach \x in {0,1,2,3,4} {
      \pgfmathsetmacro{\y}{4 - \x}
      \addplot[
      only marks,
      mark=*,
      mark size=2.5pt,
      tab-orange,
      ] coordinates {(\x,\y)};
    }

    \draw[tab-orange, ultra thick, dotted] (axis cs:-0.3, 4.3) to (axis cs:4.3, -0.3);

    % Draw arrow from origin to (x, y)
    % NOTE Could not figure out how to do this inside the loop, so
    % doing it manually
    %\draw[-Stealth, tab-orange, ultra thick] (axis cs:0,0) -- (axis cs:0,4);
    %\draw[-Stealth, tab-orange, ultra thick] (axis cs:0,0) -- (axis cs:1,3);
    %\draw[-Stealth, tab-orange, ultra thick] (axis cs:0,0) -- (axis cs:2,2);
    %\draw[-Stealth, tab-orange, ultra thick] (axis cs:0,0) -- (axis cs:3,1);
    %\draw[-Stealth, tab-orange, ultra thick] (axis cs:0,0) -- (axis cs:4,0);

    % Draw the labels for the q coordinates
    \node[anchor=north, yshift=-2pt] at (axis cs:5.25,0) {$[\vj]_1$};
    \node[anchor=east, xshift=-2pt] at (axis cs:0,5.25) {$[\vj]_2$};

    % Draw the gamma values
    \node[anchor=south west, transparentfill, xshift=2pt, yshift=2pt] at (axis cs:0,4) {$\gamma_{{\vi, (4,0)}} = \nicefrac{13}{192} $};
    \node[anchor=south west, transparentfill, xshift=2pt, yshift=2pt] at (axis cs:1,3) {$\gamma_{{\vi, (1,3)}} = -\nicefrac{1}{3} $};
    \node[anchor=south west, transparentfill, xshift=2pt, yshift=2pt] at (axis cs:2,2) {$\gamma_{{\vi, (2,2)}} = \nicefrac{5}{8} $};
    \node[anchor=south west, transparentfill, xshift=2pt, yshift=2pt] at (axis cs:3,1) {$\gamma_{{\vi, (3,1)}}$};
    \node[anchor=south west, transparentfill, xshift=2pt, yshift=2pt] at (axis cs:4,0) {$\gamma_{{\vi, (4,0)}}$};
  \end{axis}

  % Draw the values of p
  \node[anchor=north east, transparentfill] at (current bounding box.north east) {$\vi = (2, 2)$};
\end{tikzpicture}
% \end{minipage}
%%% Local Variables:
%%% mode: LaTeX
%%% TeX-master: "../main"
%%% End:

  \vspace{-1ex}
  \caption{\textbf{Illustration of \cref{eq:ttc-general} for the biharmonic operator}, \ie, the 5 values of $\vj$ with $\lVert \vj \rVert_1 = 4$ and their coefficients $\gamma_{\vi, \vj}$ to interpolate the desired mixed partials.}
  \label{fig:ttc_biharm_coeffs}
\end{wrapfigure}
This demonstrates the relevance of collapsing: it achieves a 25\,\% reduction in the quadratic coefficient.

\paragraph{Summary \& relation to other approaches for computing mixed partials.}
The scheme we propose based on \citet{griewank_evaluating_1999}'s interpolation result allows to calculate \emph{general} linear differential operators beyond Laplacians, and is amenable to collapsing.
Admittedly, \cref{eq:ttc-general} seems daunting at first glance.
However, it (i) offers a one-fits-all recipe to construct schemes for general linear PDE operators, and (ii) does not use jets of order $K'>K$ to compute $K$-th order derivatives.
It is possible to derive more ``pedagogical'' approaches, which however require hand-crafted interpolation rules case by case, and propagation of higher-order jets which is costly (see \cref{sec:appendix_ttc_other_methods} for a pedagogical example using less efficient $6$-jets to compute the biharmonic operator, or \cite[\S{}F]{shi2024stochastic} for other operators).

%%% Local Variables:
%%% mode: LaTeX
%%% TeX-master: "../main"
%%% End:
