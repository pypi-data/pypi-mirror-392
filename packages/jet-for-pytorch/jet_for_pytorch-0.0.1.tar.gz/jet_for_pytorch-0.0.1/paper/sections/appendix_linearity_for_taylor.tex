 Here, we illustrate the idea behind propagating $R$ $K$-jets through $\vf = \vg \circ \vh$ with input jets $\smash{\left(J^K\vx\right)_r(t)} = \smash{\sum_{k=0}^K \frac{t^k}{k!} \vx_{k,r}}$.
The Taylor mode scheme results from inserting \cref{eq:sum-k-directional} into \cref{eq:taylor-mode-composition}:
\begin{align}\label{eq:sum-taylor-mode-naive}
  \begin{split}
      \begin{pmatrix*}
        \vx_0
        \\
       \{\vx_{1,r}\}
        \\
        \{\vx_{2,r}\}
        \\
        \vdots
        \\
        \{\vx_{K, r}\}
      \end{pmatrix*}
      &\overset{\text{(\ref{eq:faa-di-bruno})}}{\to}
        \begin{pmatrix*}[l]
          \vh_0 =  \vh(\vx_0)
          \\
          \{\vh_{1,r}\} =
          \left\{
          \left<
          \partial \vh(\vx_0),
          \vx_{1, r}
          \right>
          \right\}
          \\
          \{\vh_{2, r}\} =
          \left\{
          \left<
          \partial^2 \vh(\vx_0),
          \vx_{1, r} \otimes \vx_{1, r}
          \right>
          +
          \left<
          \partial \vh(\vx_0),
          \vx_{2, r}
          \right>
          \right\}
          \\
          \vdots
          \\
          \{\vh_{K, r}\} =
          \left\{
          \displaystyle \sum_{
          \sigma \in \partitioning(K)
          }
          \nu(\sigma) \left<
          \partial^{|\sigma|} \vh(\vx_0),
          \tensorprod{s \in \sigma} \vx_{s, r}
          \right>
          \right\}
        \end{pmatrix*}
        \\
        &\overset{\text{(\ref{eq:faa-di-bruno})}}{\to}
        \begin{pmatrix*}[l]
          \vg_0 =  \vg(\vh_0)
          \\
          \{\vg_{1,r}\} =
          \left\{
          \left<
          \partial \vg(\vh_0),
          \vh_{1, r}
          \right>
          \right\}
          \\
          \{\vg_{2,r}\} =
          \left\{
          \left<
          \partial^2 \vg(\vh_0),
          \vh_{1, r} \otimes \vh_{1, r}\right>
          +
          \left< \partial \vg(\vh_0),
          \vh_{2, r}
          \right>
          \right\}
          \\
          \vdots
          \\
          \{\vg_{K,r}\} =
          \left\{
          \displaystyle\sum_{
          \sigma \in \partitioning(K)
          }
          \nu(\sigma) \left<
          \partial^{|\sigma|} \vg(\vh_0),
          \tensorprod{s \in \sigma} \vh_{s, r}
          \right>
          \right\}
        \end{pmatrix*}
        \\
        &\overset{\text{(\ref{eq:faa-di-bruno})}}{=}
        \begin{pmatrix*}[l]
          \vf_0 =  \vf(\vx_0)
          \\
          \{\vf_{1,r}\} = \left\{
          \left<
          \partial \vf(\vx_0),
          \vx_{1, r}
          \right>
          \right\}
          \\
          \{\vf_{2,r}\} = \left\{
          \left<
          \partial^2 \vf(\vx_0),
          \vx_{1,r} \otimes \vx_{1, r}
          \right>
          +
          \left< \partial \vf(\vx_0),
          \vx_{2, r}
          \right>
          \right\}
          \\
          \vdots
          \\
          \{\vf_{K, r}\} =
          \left\{
          \displaystyle\sum_{
          \sigma \in \partitioning(K)
          }
          \nu(\sigma) \left<
          \partial^{|\sigma|} \vf(\vx_0),
          \tensorprod{s \in \sigma} \vx_{s, r}
          \right>
          \right\}
        \end{pmatrix*}
        \\
        &\overset{\text{slice}}{\to} \{ \vg_{K,r} \}
        \\
        &\overset{\text{sum}}{\to} \sum_{r=1}^R \vg_{K,r}
          \overset{\{\vx_{1,r} = \vv_r, \vx_{2,r} =  \ldots = \vx_{K, r} = \vzero \}}{=}
          \sum_{r=1}^R \left<
          \partial^K \vf(\vx_0),
          \otimes_{k=1}^K \vv_r
          \right>
    \end{split}
\end{align}
Leveraging linearity in certain terms (in green) of the highest coefficient, as explained in \cref{eq:faa-di-bruno-expanded}, instead leads to
\begin{align}\label{eq:sum-taylor-mode-efficient}
%  \begin{split}
      \begin{pmatrix*}
        \vx_0
        \\
       \{\vx_{1,r}\}
        \\
        \{\vx_{2,r}\}
        \\
        \vdots
        \\
        \textcolor{\colorcTM}{\displaystyle\sum_{r = 1}^R\vx_{K, r}}
      \end{pmatrix*}
  &\overset{\text{(\ref{eq:faa-di-bruno})}}{\to}        \begin{pmatrix*}[l]
          \vh_0 &=  \vh(\vx_0)
          \\
          \{\vh_{1,r}\} &=
          \left\{
          \left<
          \partial \vh(\vx_0),
          \vx_{1, r}
          \right>
          \right\}
          \\
          \{\vh_{2, r}\} &=
          \left\{
          \left<
          \partial^2 \vh(\vx_0),
          \vx_{1, r} \otimes \vx_{1, r}
          \right>
          +
          \left<
          \partial \vh(\vx_0),
          \vx_{2, r}
          \right>
          \right\}
          \\
          \vdots
          \\
          \textcolor{\colorcTM}{ \displaystyle \sum_{r = 1}^R \vh_{K, r}} &=
          \displaystyle \sum_{r=1}^R \sum_{
          \sigma \in \partitioning(K) \setminus \{\tilde{\sigma}\}
          }
          \nu(\sigma) \left<
          \partial^{|\sigma|} \vh(\vx_0),
          \tensorprod{s \in \sigma} \vx_{s, r}
          \right>
          \\
          &+
          \left<
          \partial \vh(\vx_0),
          \textcolor{\colorcTM}{
          \displaystyle\sum_{r=1}^R \vx_{K, r}
          }
          \right>
        \end{pmatrix*}      \nonumber \\
       &\overset{\text{(\ref{eq:faa-di-bruno})}}{\to}
        \begin{pmatrix*}[l]
          \vg_0 &=  \vg(\vh_0)
          \\
          \{\vg_{1,r}\} &=
          \left\{
          \left<
          \partial \vg(\vh_0),
          \vh_{1, r}
          \right>
          \right\}
          \\
          \{\vg_{2,r}\} &=
          \left\{
          \left<
          \partial^2 \vg(\vh_0),
          \vh_{1, r} \otimes \vh_{1, r}\right>
          +
          \left< \partial \vg(\vh_0),
          \vh_{2, r}
          \right>
          \right\}
          \\
          \vdots
          \\
          \textcolor{\colorcTM}{\displaystyle\sum_{r=1}^R\vg_{K,r}} &=
          \displaystyle \sum_{r=1}^R \sum_{
          \sigma \in \partitioning(K) \setminus \{\tilde{\sigma}\}
          }
          \nu(\sigma) \left<
          \partial^{|\sigma|} \vg(\vh_0),
          \tensorprod{s \in \sigma} \vh_{s, r}
          \right>
          \\
          &+
          \left<
          \partial \vg(\vh_0),
          \textcolor{\colorcTM}{\displaystyle \sum_{r=1}^R\vh_{K, r}}
          \right>
        \end{pmatrix*}
         \end{align}
\begin{align*}%\label{eq:sum-taylor-mode-efficient}
  %\begin{split}
        &\overset{\text{(\ref{eq:faa-di-bruno})}}{=}
        \begin{pmatrix*}[l]
          \vf_0 &=  \vf(\vx_0)
          \\
          \{\vf_{1,r}\} &= \left\{
          \left<
          \partial \vf(\vx_0),
          \vx_{1, r}
          \right>
          \right\}
          \\
          \{\vf_{2,r}\} &= \left\{
          \left<
          \partial^2 \vf(\vx_0),
          \vx_{1, r} \otimes \vx_{1, r}
          \right>
          +
          \left< \partial \vf(\vx_0),
          \vx_{2, r}
          \right>
          \right\}
          \\
          \vdots
          \\
          \textcolor{\colorcTM}{\displaystyle \sum_{r=1}^R \vf_{K, r}} &=
          \displaystyle \sum_{r=1}^R\sum_{
          \sigma \in \partitioning(K) \setminus \{ \tilde{\sigma} \}
          }
          \nu(\sigma) \left<
          \partial^{|\sigma|} \vf(\vx_0),
          \tensorprod{s \in \sigma} \vx_{s, r}
          \right>
          \\
          &+
          \left<
          \partial \vf(\vx_0),
          \textcolor{\colorcTM}{\displaystyle \sum_{r=1}^R\vx_{K, r}}
          \right>
        \end{pmatrix*}
        \\
        &\overset{\text{slice}}{\to} \textcolor{\colorcTM}{\sum_{r=1}^R \vg_{K,r}}
          \overset{\{\vx_{1,r} = \vv_r, \vx_{2,r} =  \ldots = \vx_{K, r} = \vzero \}}{=}
          \sum_{r=1}^R \left<
          \partial^K \vf(\vx_0),
          \otimes_{k=1}^K \vv_r
          \right>
   % \end{split}
\end{align*}

\subsection{Second-order Operators: Laplacian}
Here, we show details about the propagation schemes of standard Taylor mode and collapsed Taylor mode for the computation of the Laplacian of $\vf$. We consider the decomposition $\vf = \vg \circ \vh$.

\paragraph{Standard Taylor mode.}
Using standard Taylor mode (\cref{eq:sum-taylor-mode-naive}) to compute the Laplacian yields
{
\setlength{\abovedisplayskip}{0pt}
\setlength{\belowdisplayskip}{0pt}
\begin{align}\label{eq:laplacian-naive}
  \begin{split}
    \begin{pmatrix*}
      \vx_0
      \\
      \{\vx_{1,d} \}
      \\
      \{\vx_{2,d} \}
    \end{pmatrix*}
    &\overset{\text{(\ref{eq:faa-di-bruno})}}{\to}
      \begin{pmatrix*}[l]
        \vh_0 =  \vh(\vx_0)
        \\
        \{\vh_{1,d}\} = \{
        \left<
        \partial \vh(\vx_0),
        \vx_{1,d}
        \right>
        \}
        \\
        \{\vh_{2,d}\} =
        \{
        \left<
        \partial^2 \vh(\vx_0),
        \vx_{1,d} \otimes \vx_{1,d}
        \right>
        +
        \left<
        \partial \vh(\vx_0),
        \vx_{2,d}
        \right>
        \}
      \end{pmatrix*}
    \\
    &\overset{\text{(\ref{eq:faa-di-bruno})}}{\to}
      \begin{pmatrix*}[l]
        \vg_0 =  \vg(\vh_0)
        \\
        \{\vg_{1,d}\} = \{
        \left<\partial \vg(\vh_0),
        \vh_{1,d}
        \right>
        \}
        \\
        \{\vg_{2,d}\} = \{
        \left<
        \partial^2 \vg(\vh_0),
        \vh_{1,d} \otimes \vh_{1,d}
        \right>
        +
        \left<
        \partial \vg(\vh_0),
        \vh_{2,d}
        \right>
        \}
      \end{pmatrix*}
      \\
      &\overset{\text{(\ref{eq:faa-di-bruno})}}{=}
      \begin{pmatrix*}[l]
        \vf_0 =  \vf(\vx_0)
        \\
        \{\vf_{1,d} \} = \{
        \left<
        \partial \vf(\vx_0),
        \vx_{1,d}
        \right>
        \}
        \\
        \{ \vf_{2,d} \} = \{
        \left<
        \partial^2 \vf(\vx_0),
        \vx_{1,d} \otimes \vx_{1,d}
        \right>
        +
        \left<
        \partial \vf(\vx_0),
        \vx_{2,d}
        \right>
        \}
      \end{pmatrix*}
    \\
    &\overset{\text{slice}}{\to} \{ \vg_{2,d} \}
    \\
    &\overset{\text{sum}}{\to} \sum_{d=1}^D \{ \vg_{2,d} \}
      \overset{\{\vx_{1,d} = \ve_d, \vx_{2,d} = \vzero\}}{=} \Delta \vf(\vx_0).
  \end{split}
\end{align}
}

\paragraph{Collapsed Taylor Mode AD}
Using our proposed collapsed Taylor mode, we get
{
\setlength{\abovedisplayskip}{2pt}
\setlength{\belowdisplayskip}{2pt}
\begin{align}\label{eq:laplacian-efficient}
  \begin{split}
    \begin{pmatrix*}
      \vx_0
      \\
      \{\vx_{1,d} \}
      \\
      \textcolor{\colorcTM}{\displaystyle\sum_{d=1}^D \vx_{2,d}}
    \end{pmatrix*}
    &\overset{\text{(\ref{eq:taylor-mode-scalar})}}{\to}
      \begin{pmatrix*}[l]
        \vh_0 =  \vh(\vx_0)
        \\
        \{\vh_{1,d}\} = \{
        \left<
        \partial \vh(\vx_0),
        \vx_{1,d}
        \right>
        \}
        \\
        \textcolor{\colorcTM}{\displaystyle\sum_{d=1}^D \vh_{2,d}} = \displaystyle\sum_{d=1}^D
        \left< \partial^2 \vh(\vx_0),
        \vx_{1,d} \otimes \vx_{1,d}
        \right>
        +
        \left<
        \partial \vh(\vx_0),
        \textcolor{\colorcTM}{\displaystyle\sum_{d=1}^D\vx_{2,d}}
        \right>
      \end{pmatrix*}
    \\
    &\overset{\text{(\ref{eq:taylor-mode-scalar})}}{\to}
      \begin{pmatrix*}[l]
        \vg_0 =  \vg(\vh_0)
        \\
        \{\vg_{1,d}\} = \{
        \left<
        \partial \vg(\vh_0),
        \vh_{1,d}
        \right>
        \}
        \\
        \textcolor{\colorcTM}{\displaystyle\sum_{d=1}^D\vg_{2,d}}
        =
        \displaystyle\sum_{d=1}^D
        \left<
        \partial^2 \vg(\vh_0),
        \vh_{1,d} \otimes \vh_{1,d}
        \right>
        +
        \left<
        \partial \vg(\vh_0),
        \textcolor{\colorcTM}{\displaystyle\sum_{d=1}^D\vh_{2,d}}
        \right>
      \end{pmatrix*}
      \\
      &\overset{\text{(\ref{eq:taylor-mode-scalar})}}{=}
      \begin{pmatrix*}[l]
        \vf_0 =  \vf(\vx_0)
        \\
        \{\vf_{1,d} \} = \{
        \left<
        \partial \vf(\vx_0),
        \vx_{1,d}
        \right>
        \}
        \\
        \textcolor{\colorcTM}{\displaystyle\sum_{d=1}^D \vf_{2,d}}
        =
        \displaystyle\sum_{d=1}^D
        \left<
        \partial^2 \vf(\vx_0),
        \vx_{1,d} \otimes \vx_{1,d}
        \right>
        +
        \left<
        \partial \vf(\vx_0),
        \textcolor{\colorcTM}{\displaystyle\sum_{i=1}^D\vx_{2,d}}
        \right>
      \end{pmatrix*}
    \\
    &\overset{\text{slice}}{\to} \textcolor{\colorcTM}{\sum_{d=1}^D \{ \vg_{2,d} \}}
      \overset{\{(\vx_{1,d} = \ve_d, \vx_{2,d} = \vzero)\}}{=}
      \Delta \vf(\vx_0)
  \end{split}
\end{align}
}

%%% Local Variables:
%%% mode: LaTeX
%%% TeX-master: "../main"
%%% End:
