Using neural networks to learn functions constrained by physical laws is a popular trend in scientific machine learning \cite{carleo2017solving,pfau2020ab,hermann2020deep,hu2024hutchinson,karniadakis2021physics, raissi2019physics,sun2020global}.
Typically, the Physics is encoded through partial differential equations (PDEs) that the neural net must satisfy.
The associated loss functions require evaluating differential operators \wrt the net's input, rather than weights.
Evaluating these differential operators remains a computational challenge, especially if they contain high-order derivatives.

\paragraph{Computing PDE operators.} Two important fields that build on PDE operators are variational Monte-Carlo (VMC) simulations and Physics-informed neural networks (PINNs).
VMC employs neural networks as ansatz for the Schr\"odinger equation \cite{carleo2017solving, pfau2020ab, hermann2020deep} and demands computing the net's Laplacian (the Hessian trace) for the Hamiltonian's kinetic term.
PINNs represent PDE solutions as a neural net and train it by minimizing the residuals of the governing equations \cite{raissi2019physics, karniadakis2021physics}. For instance, Kolmogorov-type equations like the Fokker-Planck and Black-Scholes equation require weighted second-order derivatives on high-dimensional spatial domains \cite{hu2024hutchinson, sun2024dynamical}.
Other PINNs for elasticity problems use the biharmonic operator \cite{vahab_physics-informed_2022, hu2024hutchinson, vikas_biharm, shi2024stochastic}, which contains fourth-order derivatives.

\paragraph{Is backpropagation all we need?}
% Alternative: The gap between theory and practice
Although nesting first-order automatic differentiation (AD) to compute high-order derivatives scales exponentially \wrt the degree in time and memory \cite[][\S3.2]{shi2024stochastic}, this approach is common practice: it is easy to implement in ML libraries, and their backpropagation is highly optimized.
A promising alternative is \emph{Taylor mode AD}~\cite[or simply \emph{Taylor mode},][\S13]{griewank2008evaluating}, introduced to the ML community in \citeyear{bettencourt2019taylor}, which scales polynomially \wrt the degree in time and memory \cite{griewank_evaluating_1999}.
However, we observe empirically that vanilla Taylor mode may not be enough to beat nesting (\cref{fig:vanilla-taylor-not-enough}): evaluating the Laplacian of a 5-layer MLP, using JAX's \emph{Taylor mode is 50\% slower than nested backpropagation} that computes, then traces, the Hessian via Hessian-vector products \cite{pearlmutter1994fast}.
This calls into question the relevance of Taylor mode for computing common PDE operators.

\paragraph{The advent of forward schemes.}
Recent works have successfully demonstrated the potential of modified forward propagation schemes, though.
For the Laplacian, \citet{li2023forward, li2024dof} developed a special forward propagation framework called the \emph{forward Laplacian}, whose JAX implementation \cite{gao2023folx} is \emph{roughly twice as fast as nested first-order AD} (\cref{fig:vanilla-taylor-not-enough}).
While the forward Laplacian does not rely on Taylor mode, recent work pointed out a connection \cite{dangel2024kroneckerfactored}; it remains unclear, though, if efficient forward schemes exist for other differential operators, and how they relate to Taylor mode.
Concurrently, \citet{shi2024stochastic} derived stochastic approximations of differential operators in high dimensions by evaluating Taylor mode along suitably sampled random directions.

Irrespective of stochastic or exact computation, at their core, these popular PDE operators are \emph{linear}: we must evaluate derivatives along multiple directions, then sum them.
Based on this linearity, we identify an optimization technique to rewrite the computational graph of standard Taylor mode that is applicable to general linear PDE operators and randomized Taylor mode:

\begin{figure*}[!t]
  \centering
  \begin{minipage}[b]{0.42\linewidth}
    \centering
    \newsavebox{\legend}
\savebox{\legend}{
  \begin{tikzpicture}[font=\small]
    \matrix [%
    matrix of nodes,%
    ampersand replacement=\&,% to use inside a savebox
    nodes={anchor=west, align=left, inner sep=1pt},%
    column sep=1ex,%
    row sep=0ex,%
    ] (legend)
    {
      \draw[tab-blue] plot[mark=*] coordinates {(0,0)};
      \& \node[align=left, anchor=west, tab-blue, inner sep=1pt]{Nested 1\textsuperscript{st}-\\ order AD};
      \\[0.5ex]
      \draw[tab-orange] plot[mark=triangle*, rotate=270] coordinates {(0,0)};
      \& \node[align=left, anchor=west, tab-orange, inner sep=1pt]{Vanilla Tay-\!\\ lor mode};
      \\[0.5ex]
      \draw[tab-green] plot[mark=triangle*, rotate=90] coordinates {(0,0)};
      \& \node[align=left, anchor=west, tab-green, inner sep=1pt]{Forward\\ Laplacian};
      \\
    };
  \end{tikzpicture}
}

\begin{tikzpicture}
  \node (image) [inner sep=0pt] {% [trim={left bottom right top},clip]
    \includegraphics[trim={0 4cm 0 0},clip]{jet/exp/exp04_jax_benchmark/figures/architecture_tanh_mlp_768_768_512_512_1_device_cuda_dim_50_name_jax_laplacian_vary_batch_size.pdf}};
  \node (xaxis) [inner sep=0pt, anchor=north] at (image.south) {% [trim={left bottom right top},clip]
    \includegraphics[trim={0 0 0 6.4cm},clip]{jet/exp/exp04_jax_benchmark/figures/architecture_tanh_mlp_768_768_512_512_1_device_cuda_dim_50_name_jax_laplacian_vary_batch_size.pdf}};

  \node[rectangle, draw=gray, rounded corners, inner sep=-1pt, anchor=north west, xshift=1.35cm, fill=white, fill opacity=0.8, text opacity=1.0] at (image.north west) {\usebox{\legend}};
\end{tikzpicture}

%%% Local Variables:
%%% mode: LaTeX
%%% TeX-master: "../main"
%%% End:


    \caption{\textbf{$\blacktriangle$ Vanilla Taylor mode is not enough to beat nested 1\textsuperscript{st}-order AD.}
      Illustrated for computing the Laplacian of a $\mathrm{tanh}$-activated $50 \!\to\! 768 \!\to\! 768 \!\to\! 512 \!\to\! 512 \!\to\! 1$
      MLP with JAX (+ \texttt{jit}) on GPU (details in \Cref{sec:jax-benchmark}).
      We show how to automatically obtain the specialized forward Laplacian through simple graph transformations that ``collapse`` vanilla Taylor mode.
    }\label{fig:vanilla-taylor-not-enough}

    \vspace{0.25ex}
    \caption{\textbf{$\blacktriangleright$ Collapsed Taylor mode directly propagates the sum of highest degree coefficients.}
      Visualized for pushing 4 $K$-jets through a $\sR^5 \!\to\! \sR^3 \!\to\! \sR$ function ($K=2$ yields the forward Laplacian).
      %\Cref{sec:background} introduces the notation.
      }\label{fig:visual-abstract}
  \end{minipage}
  \hfill
  \begin{minipage}[b]{0.57\linewidth}
    \centering
    \newcommand{\drawgridrectangle}[4]{%
  \begin{tikzpicture}[scale=#4]
    \pgfmathsetmacro{\ymax}{#1}
    \pgfmathsetmacro{\xmax}{#2}

    % Fill the rectangle
    \fill[#3] (0,0) rectangle (\xmax,\ymax);

    % Draw the border
    \draw[white, line width=#4*3pt] (0,0) rectangle (\xmax,\ymax);

    % Draw vertical grid lines
    \pgfmathsetmacro{\xsteps}{#2}
    \foreach \x in {1,...,\xsteps} {
      \draw[white, line width=#4*3pt] (\x,0) -- (\x,\ymax);
    }

    % Draw horizontal grid lines
    \pgfmathsetmacro{\ysteps}{#1}
    \foreach \y in {1,...,\ysteps} {
      \draw[white, line width=#4*7pt] (0,\y) -- (\xmax,\y);
    }
  \end{tikzpicture}%
}

\centering
\newsavebox{\visualAbstract}
\savebox{\visualAbstract}{
  \begin{tikzpicture}
    % ===================================================================
    % DRAW VECTORS OF DERIVATIVES 0, 1, ..., K
    % ===================================================================
    \matrix [%
    matrix of nodes,%
    ampersand replacement=\&,% to use inside a savebox
    nodes={anchor=center, align=center},%
    column sep=5ex,%
    row sep=1ex,%
    ] (taylor)
    {
      \drawgridrectangle{1}{5}{gray!25!white}{0.33}
      \& \drawgridrectangle{1}{3}{gray!25!white}{0.33}
      \& \drawgridrectangle{1}{1}{gray!25!white}{0.33}
      \\[-1.5ex]
      $\vx_0$ \& $\vh_0$ \& $\vg_0$
      \\
      \drawgridrectangle{4}{5}{gray!50!white}{0.33}
      \& \drawgridrectangle{4}{3}{gray!50!white}{0.33}
      \& \drawgridrectangle{4}{1}{gray!50!white}{0.33}
      \\[-1.5ex]
      $\{\vx_{1, r}\}$ \& $\{\vh_{1, r}\}$ \& $\{\vg_{1, r}\}$
      \\[0.5ex]
      \drawgridrectangle{1}{5}{white}{0.33}
      \& \drawgridrectangle{1}{3}{white}{0.33}
      \& \drawgridrectangle{1}{1}{white}{0.33}
      \\[0.5ex]
      \\
      \drawgridrectangle{4}{5}{gray!75!white}{0.33}
      \& \drawgridrectangle{4}{3}{gray!75!white}{0.33}
      \& \drawgridrectangle{4}{1}{gray!75!white}{0.33}
      \\[-1.5ex]
      $\{\vx_{K-1, r}\}$ \& $\{\vh_{K-1, r}\}$ \& $\{\vg_{K-1, r}\}$
      \\[0.5ex]
      \drawgridrectangle{4}{5}{tab-orange}{0.33}
      \&
      \drawgridrectangle{4}{3}{tab-orange}{0.33}
      \&
      \drawgridrectangle{4}{1}{tab-orange}{0.33}
      \\[-1.5ex]
      $\{\vx_{K, r}\}$ \& $\{\vh_{K,r}\}$ \& $\{\vg_{K,r}\}$
      \\
      \drawgridrectangle{1}{5}{tab-green}{0.33}
      \& \drawgridrectangle{1}{3}{tab-green}{0.33}
      \& \drawgridrectangle{1}{1}{tab-green}{0.33}
      \\[-1ex]
      \textcolor{tab-green}{$\sum_r \vx_{K,r}$}
      \& \textcolor{tab-green}{$\sum_r \vh_{K,r}$}
      \& \textcolor{tab-green}{$\sum_r \vg_{K,r}$}
      \\
    };

    \node at (taylor-5-1) {\vdots};
    \node at (taylor-5-2) {\vdots};
    \node at (taylor-5-3) {\vdots};

    % ===================================================================
    % DRAW DEPENDENCY ARROWS
    % ===================================================================
    \pgfmathsetmacro{\K}{5}
    \pgfmathsetmacro{\L}{2}

    \foreach \l in {1,...,\L}{
      \pgfmathsetmacro{\lother}{int(\l+1)}
      \foreach \k in {1,...,\K} {
        \pgfmathsetmacro{\row}{int(2*\k-1)}
        \foreach \kother in {\k,...,\K} {
          \pgfmathsetmacro{\rowother}{int(2*\kother-1)}
          \draw[-Stealth, line width=1pt, gray!50!white] (taylor-\row-\l.east) -- (taylor-\rowother-\lother.west);
        }
      }
    }

    % ===================================================================
    % DRAW ARROW INDICATING DIRECTION OF PROPAGATION
    % ===================================================================
    \coordinate (arrowStart) at ($(taylor-1-1.north)+(0,3.5ex)$);
    \coordinate (arrowEnd) at ($(taylor-1-3.north east)+(0,3.5ex)$);
    \draw[-Stealth, line width=2pt, black] (arrowStart) to node [midway, fill=white, align=center] {\textbf{Taylor forward} \\ \textbf{propagation}} (arrowEnd);

    % ===================================================================
    % DRAW DERIVATIVE DEGREE LABELS
    % ===================================================================
    \node [minimum width=8ex, align=center, left=1.5ex of taylor-1-1] (zero) {0};
    \node [minimum width=8ex, align=center, left=1.5ex of taylor-3-1] {1};
    \node [minimum width=8ex, align=center, left=1.5ex of taylor-5-1] {$\vdots$};
    \node [minimum width=8ex, align=center, left=1.5ex of taylor-7-1] {$K-1$};
    \node [minimum width=8ex, align=center, left=1.5ex of taylor-9-1] {$K$};
    \node [minimum width=8ex, align=center, left=1.5ex of taylor-11-1, yshift=-8pt] {$K$ \\ \textbf{(ours)}};

    \node [align=center] (coefficientLabel) at ($(zero)+(0, 5.5ex)$) {\textbf{Derivative}\\\textbf{degree}};

    % ===================================================================
    % DRAW THE RED BOX HIGHLIGHTING THE HIGHEST COMPONENT
    % ===================================================================
    \draw[rounded corners, line width=2pt, red] (taylor-9-1.north west) rectangle (taylor-10-3.south east);
    \draw[white, fill=white] ($(taylor-10-3.south east)+(-0.3ex, 2ex)$) rectangle ($(taylor-10-3.south east)+(0.3ex, 11.25ex)$);

    \draw [-Stealth, line width=1.25pt, red]
    ($(taylor-10-3.south east)+(0, 2.75ex)$)
    -- ($(taylor-10-3.south east)+(0, 5.75ex)$);

    \draw [-Stealth, line width=1.25pt, red]
    ($(taylor-10-3.south east)+(0, 10.5ex)$)
    -- ($(taylor-10-3.south east)+(0, 7.5ex)$);
    ;

    \draw [red, line width=5pt]
    ($(taylor-10-3.south east)+(-0.7ex, 6.625ex)$)
    -- ($(taylor-10-3.south east)+(0.7ex, 6.625ex)$);
  \end{tikzpicture}
}

\resizebox{\linewidth}{!}{
  \begin{tikzpicture}
    \node {\usebox{\visualAbstract}};
  \end{tikzpicture}
}

%%% Local Variables:
%%% mode: LaTeX
%%% TeX-master: "../main"
%%% End:

  \end{minipage}
\end{figure*}

\begin{enumerate}[leftmargin=0.5cm]
\item \textbf{We propose optimizing standard Taylor mode by collapsing the highest Taylor coefficients,} directly \textcolor{tab-green}{\bfseries propagating their sum}, rather than \textcolor{tab-orange}{\bfseries propagating then summing} (\cref{fig:visual-abstract}).
Our approach contains the forward Laplacian as special case, is applicable to randomized Taylor mode, and also general linear PDE operators, which we show using the techniques from \citet{griewank_evaluating_1999}.

%and, relying on [...] can be generalized to 

  %Collapsing standard Taylor mode for the Laplacian yields the forward Laplacian \cite{li2023forward}, but we show that this optimization is applicable to many other differential operators, and stochastic Taylor mode \cite{shi2024stochastic}.

\item \textbf{We show how to collapse standard Taylor mode by simple graph rewrites based on linearity.}
  This leads to a clean separation of concepts:
  Users can build their computational graph using standard Taylor mode, then rewrite it to collapse it.
  Due to the simple nature of our proposed rewrites, they could easily be absorbed into the just-in-time (JIT) compilation of ML frameworks without introducing a new interface or exposing complexity to users.

\item \textbf{We empirically demonstrate that collapsing Taylor mode accelerates standard Taylor mode.}
We implement a Taylor mode library\footnote{Available at \url{https://github.com/f-dangel/torch-jet}.} for PyTorch \cite{paszke2019pytorch} that realizes the graph simplifications with \texttt{torch.fx} \cite{reed2022torch}.
On popular PDE operators, we empirically find that, compared to standard Taylor mode, collapsed Taylor mode achieves superior performance that is well-aligned with the theoretical expectation, while consistently outperforming nested first-order AD.
\end{enumerate}

Our work takes an important step towards the broader adoption of Taylor mode as viable alternative to nested first-order AD for computing PDE operators, while being as easy to use. 

%%% Local Variables:
%%% mode: LaTeX
%%% TeX-master: "../main"
%%% End:
