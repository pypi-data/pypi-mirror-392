In this section, we illustrate the two graph simplifications that are required to collapse Taylor mode.

We will consider collapsing the 2-jet of $f = \sin$ as an example.
Recall the propagation scheme \cref{eq:sum-taylor-mode-naive} and assume that the Taylor coefficients are given by $\{\vx_{0,r} = \vx_0\}$, $\{\vx_{1,r}\}$, and $\{\vx_{2,r}\}$ where $r$ indexes the directions along which we evaluate the sum:
\begin{align*}
  \begin{matrix}
    \vx_0
    \\
    \{\vx_{1,r}\}
    \\
    \{\vx_{2,r}\}
  \end{matrix}
  &\overset{\text{replicate $\vx_0$}}{\to}
    \begin{Bmatrix}
      \vx_{0,r} = \vx_0
      \\
      \vx_{1,r}
      \\
      \vx_{2,r}
    \end{Bmatrix}
    \overset{\eqref{eq:sum-taylor-mode-naive}}{\to}
    \begin{Bmatrix}
      \vf_{0,r} = \sin(\vx_0)
      \\
      \vf_{1,r} = \cos(\vx_0) \odot \vx_{1,r}
      \\
      \vf_{2,r} = -\sin(\vx_0) \odot \vx_{1,r} \odot \vx_{1, r} + \cos(\vx_0) \odot \vx_{2,r}
    \end{Bmatrix}
  \\
  &\overset{\text{sum highest component}}{\to}
    \begin{matrix}
      \begin{Bmatrix}
        \vf_{0,r}
        \\
        \vf_{1,r}
      \end{Bmatrix}
      \\
      \sum_r \vf_{2,r}
    \end{matrix}
\end{align*}
Here, $\sin$ applies element-wise and $\odot$ denotes element-wise multiplication.
The computational graph for this procedure is displayed in the following diagram, with input and output nodes highlighted in dark and light gray. The suffix \texttt{\_r} means that all $R$ corresponding tensors are stacked along their leading axis.
$\texttt{replicate}$ is a function that replicates a tensor $R$ times along a new leading axis, which is in PyTorch usually for free and without additional memory overhead (using \texttt{torch.expand}). All other functions refer to those of the PyTorch API:

\begin{figure}[!h]
  \centering
  \scalebox{0.66}{{
  \tikzset{state/.style={rectangle, rounded corners, align=left, font=\ttfamily, draw=black, anchor=west, align=left, inner sep=5pt}}
  \tikzset{leaf/.style={fill=gray!50!white}}
  \tikzset{output/.style={fill=gray!25!white}}
  \begin{tikzpicture}
    % draw the nodes
    \matrix [%
    ampersand replacement=\&,% to use inside a savebox
    column sep=5ex,%
    row sep=4ex,%
    ]{
      \node[state, leaf] (x0) {x0};
      \& \node[state] (x0-d) {x0\_r = replicate(x0)};
      \&
      \& \node[state, output] (f0-d) {f0\_r = sin(x0\_r)};
      \\
      \node[state, leaf] (x1-d) {x1\_r};
      \& \node[state] (d2f-d) {d2f\_r = -f0\_r};
      \& \node[state] (df-d) {df\_r = cos(x0\_r)};
      \& \node[state, output] (f1-d) {f1\_r =
        einsum(``...,...\\->...'', df\_r, x1\_r)};
      \\
      \&
      \& \node[state] (temp1-d) {temp1\_r = einsum(``...,...,...\\->...'', df\_r, x1\_r, x1\_r)};
      \\
      \node[state, leaf] (x2-d) {x2\_r};
      \& \node[state] (temp2-d) {temp2\_r = einsum(``...,...\\->...'', d2f\_r, x2\_r)};
      \& \node[state] (f2-d) {f2\_r = temp1\_r + temp2\_r};
      \& \node[state, output] (f2) {f2 = sum(f2\_r, dim=0)};
      \\
    };

    % draw the relations
    \tikzset{arrow/.style={-Stealth}}
    \path
    (x0) edge [arrow] (x0-d)
    (x0-d) edge [arrow] (f0-d)
    (x0-d) edge [arrow] (df-d)
    (f0-d) edge [arrow] (d2f-d)
    (df-d) edge [arrow] (f1-d)
    (x1-d) edge [arrow, out=18, in=175] (f1-d)
    (d2f-d) edge [arrow] (temp2-d)
    (df-d) edge [arrow, out=-60, in=100] (temp1-d)
    (x2-d) edge [arrow] (temp2-d)
    (x1-d) edge [arrow, out=-25,in=170] (temp1-d)
    (temp1-d) edge [arrow] (f2-d)
    (temp2-d) edge [arrow] (f2-d)
    (f2-d) edge [arrow] (f2)
    ;
  \end{tikzpicture}
}
%%% Local Variables:
%%% mode: LaTeX
%%% TeX-master: "../main"
%%% End:
}
\end{figure}

Our simplification proceeds in two steps.
First, propagate \texttt{replicate} nodes down the graph to remove repeated computations on the same tensors. This is done in a forward traversal through the graph.
Second, in a single backward traversal through the graph, we propagate the \texttt{sum} node up.
After applying both steps, the graph looks as follows:

\begin{figure}[!h]
  \centering
  \scalebox{0.66}{
    \input{figures/sin_2jet_9.tex}
  }
\end{figure}

Two important properties of the new graph are (i) the \texttt{replicate} node moved to an output node, hence the corresponding redundant computation was successfully removed (ii) the highest component \texttt{x2\_r} is immediately summed then propagated, \ie, we collapsed Taylor mode and avoid the separate propagation for all \texttt{x2\_r}.

We will now illustrate the two simplification steps in full detail.
The first stage starts from the original graph and pushes forward the replicate node, as illustrated step-by-step in \cref{fig:push-replicate-simplification}.
The second stage starts from the graph produced by the replicate-push procedure, and propagates the final sum node up the graph, illustrated by \cref{fig:pull-sum-simplification}.
This yields the final computation graph shown above.

\begin{figure}[!h]
  \scalebox{0.70}{
    \input{figures/sin_2jet_1.tex}
  }
  \vspace{-0.6ex}
  \noindent\rule{\textwidth}{1pt}
  \vspace{-0.6ex}

  \scalebox{0.70}{
    \input{figures/sin_2jet_2.tex}
  }

  \vspace{-0.6ex}
  \noindent\rule{\textwidth}{1pt}
  \vspace{-0.6ex}

  \scalebox{0.70}{
    \input{figures/sin_2jet_3.tex}
  }

  \vspace{-0.6ex}
  \noindent\rule{\textwidth}{1pt}
  \vspace{-0.6ex}
  \scalebox{0.70}{
    {
  \tikzset{state/.style={rectangle, rounded corners, align=left, font=\ttfamily, draw=black, anchor=west, align=left, inner sep=5pt}}
  \tikzset{leaf/.style={fill=gray!50!white}}
  \tikzset{output/.style={fill=gray!25!white}}
  \tikzset{replicate/.style={fill=blue!25!white}}
  \begin{tikzpicture}
    % draw the nodes
    \matrix [%
    ampersand replacement=\&,% to use inside a savebox
    column sep=5ex,%
    row sep=4ex,%
    ]{
      \node[state, leaf] (x0) {x0};
      \&
      \& \node[state] (f0) {f0 = sin(x0)};
      \& \node[state, replicate] (f0-d) {f0\_r = replicate(f0)};
      \\
      \node[state, leaf] (x1-d) {x1\_r};
      \& \node[state] (d2f-d) {d2f = -f0};
      \& \node[state] (df) {df = cos(x0)};
      \& \node[state, output] (f1-d) {f1\_r =
        einsum(``...,r...\\->r...'', df, x1\_r)};
      \\
      \&
      \& \node[state, replicate] (temp1-d) {temp1\_r = einsum(``...,r...,r...\\->r...'', df, x1\_r, x1\_r)};
      \\
      \node[state, leaf] (x2-d) {x2\_r};
      \& \node[state, replicate] (temp2-d) {temp2\_r = einsum(``...,r...\\->r...'', d2f, x2\_r)};
      \& \node[state] (f2-d) {f2\_r = temp1\_r + temp2\_r};
      \& \node[state, output] (f2) {f2 = sum(f2\_r, dim=0)};
      \\
    };

    % draw the relations
    \tikzset{arrow/.style={-Stealth}}
    \path
    (x0) edge [arrow] (f0)
    (f0) edge [arrow] (f0-d)
    (x0) edge [arrow, out=-30, in=170] (df)
    (f0) edge [arrow] (d2f-d)
    (df) edge [arrow] (f1-d)
    (x1-d) edge [arrow, out=18, in=175] (f1-d)
    (d2f-d) edge [arrow] (temp2-d)
    (df) edge [arrow, out=-60, in=100] (temp1-d)
    (x2-d) edge [arrow] (temp2-d)
    (x1-d) edge [arrow, out=-25,in=170] (temp1-d)
    (temp1-d) edge [arrow] (f2-d)
    (temp2-d) edge [arrow] (f2-d)
    (f2-d) edge [arrow] (f2)
    ;
  \end{tikzpicture}
}
%%% Local Variables:
%%% mode: LaTeX
%%% TeX-master: "../main"
%%% End:

      }

  \caption{\textbf{Step-by-step illustration of pushing \texttt{replicate} nodes down a computation graph.}}
  \label{fig:push-replicate-simplification}
\end{figure}
%%% Local Variables:
%%% mode: LaTeX
%%% TeX-master: "../main"
%%% End:


\begin{figure}[!h]
  \centering

  \scalebox{0.70}{
    \input{figures/sin_2jet_5.tex}
  }

  \vspace{-0.6ex}
  \noindent\rule{\textwidth}{1pt}
  \vspace{-0.6ex}

  \scalebox{0.70}{
    \input{figures/sin_2jet_6.tex}
  }

  \vspace{-0.6ex}
  \noindent\rule{\textwidth}{1pt}
  \vspace{-0.6ex}

  \scalebox{0.70}{
    \input{figures/sin_2jet_7.tex}
  }


  \vspace{-0.6ex}
  \noindent\rule{\textwidth}{1pt}
  \vspace{-0.6ex}

  \scalebox{0.70}{
    {
  \tikzset{state/.style={rectangle, rounded corners, align=left, font=\ttfamily, draw=black, anchor=west, align=left, inner sep=5pt}}
  \tikzset{leaf/.style={fill=gray!50!white}}
  \tikzset{output/.style={fill=gray!25!white}}
  \tikzset{sum/.style={fill=green!25!white}}
  \begin{tikzpicture}
    % draw the nodes
    \matrix [%
    ampersand replacement=\&,% to use inside a savebox
    column sep=5ex,%
    row sep=4ex,%
    ]{
      \node[state, leaf] (x0) {x0};
      \&
      \& \node[state] (f0) {f0 = sin(x0)};
      \& \node[state, output] (f0-d) {f0\_r = replicate(f0)};
      \\
      \node[state, leaf] (x1-d) {x1\_r};
      \& \node[state] (d2f-d) {d2f = -f0};
      \& \node[state] (df) {df = cos(x0)};
      \& \node[state, output] (f1-d) {f1\_r = einsum(``...,r...\\->r...'', df, x1\_r)};
      \\
      \&
      \& \node[state, sum] (temp1) {temp1 = einsum(``...,r...,r...\\->...'', df, x1\_r, x1\_r)};
      \\
      \node[state, leaf] (x2-d) {x2\_r};
      \& \node[state, sum] (x2) {sum(x2\_r, dim=0)};
      \& \node[state] (temp2) {temp2 = einsum(``...,r...\\->...'', d2f, x2)};
      \& \node[state, output] (f2) {f2 = temp1 + temp2};
      \\
    };

    % draw the relations
    \tikzset{arrow/.style={-Stealth}}
    \path
    (x0) edge [arrow] (f0)
    (f0) edge [arrow] (f0-d)
    (x0) edge [arrow, out=-30, in=170] (df)
    (f0) edge [arrow] (d2f-d)
    (df) edge [arrow] (f1-d)
    (x1-d) edge [arrow, out=18, in=175] (f1-d)
    (d2f-d) edge [arrow, out=-30, in=168] (temp2)
    (df) edge [arrow, out=-50, in=100] (temp1)
    (x2-d) edge [arrow] (x2)
    (x2)   edge [arrow] (temp2)
    (x1-d) edge [arrow, out=-25,in=170] (temp1)
    (temp1) edge [arrow] (f2)
    (temp2) edge [arrow] (f2)
    ;
  \end{tikzpicture}
}
%%% Local Variables:
%%% mode: LaTeX
%%% TeX-master: "../main"
%%% End:

  }

  \caption{\textbf{Step-by-step illustration of propagating \texttt{sum} nodes up a computation graph.}}
  \label{fig:pull-sum-simplification}
\end{figure}
%%% Local Variables:
%%% mode: LaTeX
%%% TeX-master: "../main"
%%% End:


%%% Local Variables:
%%% mode: LaTeX
%%% TeX-master: "../main"
%%% End:
