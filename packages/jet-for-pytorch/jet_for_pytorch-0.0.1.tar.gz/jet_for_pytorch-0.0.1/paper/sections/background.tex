Taylor mode AD (or, simply, Taylor mode) computes higher-order derivatives---as needed, \eg, for PDE operators---through propagation of Taylor coefficients according to the chain rule.

\paragraph{Scalar case.}
To illustrate Taylor mode, consider the scalar function $f: \sR \to \sR$ and extend the input variable $x$ to a path $x(t)$ with $x(0) = x_0$, whose form is a univariate Taylor polynomial of degree $K$, $\smash{x(t) = \sum_{k=0}^K \frac{t^k}{k!} x_k}$ with $x_k$ the $k$-th Taylor coefficient.
If $f$ is smooth enough, we can evaluate Taylor coefficients of the transformed path $\smash{f(x(t)) = \sum_{k=0}^K \frac{t^k}{k!} f_k}$ with $\smash{f_k \coloneqq \frac{\mathrm{d}^k}{\mathrm{d}t^k} f(x(t)) |_{t=0}}$.
The chain rule provides the coefficients' propagation rules.
\Eg, for degree $K=3$ we get
\begin{align}
  \label{eq:taylor-mode-scalar}
  \begin{matrix*}[l]
    f_0 = f(x_0)\,,
    \\[0.75ex]
    f_1 = \partial f(x_0) x_1\,,
  \end{matrix*}
  \qquad
  \begin{matrix*}[l]
    f_2 = \partial^2 f(x_0)x_1^2 + \partial f(x_0) x_2\,,
    \\[0.75ex]
    f_3
    = \partial^3 f(x_0)x_1^3 + 3 \partial^2 f(x_0) x_1 x_2 + \partial f(x_0) x_3\,.
  \end{matrix*}
\end{align}
\citet{faa1857note} provided the general formula for $f_k$, and \citet{fraenkel1978formulae} extended it to the multivariate case \cite[see also][]{arbogast1800calcul,hardy2006combinatorics}.
It serves as foundation for Taylor mode to compute higher-order derivatives \citep[\eg,][\S13]{griewank2008evaluating}:
setting $x_1 = 1, x_2 = x_3 = 0$ yields $f_1 = \partial f(x_0), f_2 = \partial^2 f(x_0), f_3 = \partial^3 f(x_0)$.
We call the univariate Taylor polynomial of a function $x(t)$ of degree $K$, represented by the coefficients $(x_0, \dots, x_K)$, the \emph{$K$-jet of $x$}, following the terminology of JAX's Taylor mode \cite{bettencourt2019taylor}.

\paragraph{Notation for multivariate case.}
We consider the general case of computing higher-order derivatives, \eg, PDE operators, of a vector-to-vector function $\vf: \sR^D \to \sR^C$.
This requires additional notation to generalize \cref{eq:taylor-mode-scalar}.
Given $K$ vectors $\vv_1, \dots, \vv_K \in \sR^D$, we write their tensor product as
\begin{equation*}
  \otimes_{k=1}^K \vv_k = \vv_1 \otimes \ldots \otimes \vv_K
  \in ( \sR^D )^{\otimes K}
  \quad
  \text{with entries}
  \quad
  \left[\otimes_{k=1}^K \vv_k\right]_{d_1, \dots, d_K}
  = [\vv_1]_{d_1} \cdot \ldots \cdot [\vv_K]_{d_K}
\end{equation*}
for $d_1, \dots, d_K \in \{1, \dots, D\}$, and compactly write $\vv^{\otimes K} = \otimes_{k=1}^K \vv$.
We define the inner product of two tensors $\smash{\tA, \tB \in (\sR^{D})^{\otimes K}}$ as the Euclidean inner product of their flattened versions
\begin{align}\label{eq:derivative-tensor-scalar-product}
  \left\langle
  \tA, \tB
  \right\rangle
  \coloneqq
  \sum_{d_1}
  \sum_{d_2}
  \dots
  \sum_{d_K}
  \left[\tA\right]_{d_1, d_2, \dots, d_K}
  \left[\tB\right]_{d_1, d_2, \dots, d_K} \in \sR\,.
\end{align}
We allow broadcasting in \cref{eq:derivative-tensor-scalar-product}: if one tensor has more dimensions but matching trailing dimensions, we take the inner product for each component of the leading dimensions.
This allows to express contractions with derivative tensors of vector-valued functions, \eg,
contracting the $k$-th derivative tensor $\partial^k \vf(\vx_0) \in \sR^C \times (\sR^D)^{\otimes k}$, such that $\langle \tA, \partial^k \vf(\vx_0) \rangle \in \sR^C$.

\paragraph{Multivariate case \& composition.}
Evaluating the $K$-jet of $\vf$ at $\vx_0 \in \sR^D$ starts with the extension of $\vx_0$ to a smooth path $\vx: \sR \to \sR^D$ with $\vx(0) = \vx_0$.
Formally, the $K$-jet of $\vf$ is defined as
\begin{align*}
  J^K \vf : \sR \to \sR^C\,,
  \quad (J^K \vf)(t) := \sum_{k=0}^K \frac{t^k}{k!} \vf_k
  \quad \text{with} \quad
  \vf_k := \frac{\mathrm{d}^k}{\mathrm{d}t^k} \vf(\vx(t)) |_{t=0}
\end{align*}
and requires the $K$-jet of $\vx$, $(J^K \vx)(t) := \sum_{k=0}^K \frac{t^k}{k!} \vx_k$.
As we are interested in the coefficients, we will slightly abuse the $K$-jet as mapping $(\vx_0, \dots, \vx_K) \mapsto (\vf_0, \dots, \vf_K)$  (see \cref{fig:utp} for an illustration).

As is common for AD, propagating the coefficients is broken down into
composing $\vf$ of atomic functions with known derivatives and the chain rule.
In the simplest case, let $\vf = \vg \circ \vh: \sR^D \to \sR^I \to \sR^C$ for two elemental functions $\vg, \vh$.
Given the input $K$-jet for $\vx$, the coefficients $\vh_k = \smash{\frac{\mathrm{d}^k}{\mathrm{d}t^k}\vh(\vx(t))} |_{t=0}$ follow from the generalized Faà di Bruno formula (spelled out for some $k$s in \cref{sec:faa-di-bruno-cheatsheet})
\begin{align}
  \label{eq:faa-di-bruno}
  % \textstyle % Comment out this line if we have enough space
  \vh_k
  =
  \sum_{\sigma \in \partitioning(k)}
  \nu(\sigma)
  \left<
  \partial^{|\sigma|} \vh,
  \tensorprod{s \in \sigma} \vx_s
  \right>
  \quad
  \text{with}
  \quad
  \nu(\sigma)
  =
  \frac{k!}{
    \left(
      \prod_{s \in \sigma
      }
      n_s!
    \right)
    \left(
      \prod_{s \in \sigma}
      s!
    \right)
  }\,.
\end{align}
Here, $\partitioning(k)$ is the integer partitioning of $k$ (a set of sets), $\nu$ is a multiplicity function, and $n_s$ counts occurrences of $s$ in a set $\sigma$ (\eg, $n_1(\{1,1,3\})\!=\!2$ and $n_3 \!= \!1$).
Propagating the $\vh_k$s through $\vg$ results in the $K$-jet for $\vf$.
In summary, the propagation scheme is (with $\vx_k \in \sR^D$, $\vh_k \in \sR^I$, $\vf_k \in \sR^C$)
\begin{align}\label{eq:taylor-mode-composition}
  \begin{split}
    &\begin{pmatrix*}
      \vx_0
      \\
      \vx_1
      \\
      \vx_2
      \\
      \vdots
      \\
      \vx_K
    \end{pmatrix*}
      \overset{\text{(\ref{eq:faa-di-bruno})}}{\to}
      \begin{pmatrix*}[l]
        \vh_0 \!=\!  \vh(\vx_0)
        \\
        \vh_1 \!=\!  \left<
        \partial \vh(\vx_0),
        \vx_1
        \right>
        \\
        \vh_2 \!=\! \left<
        \partial^2 \vh(\vx_0),
        \vx_1^{\otimes 2}
        \right>
        \!+\!
        \left <
        \partial \vh(\vx_0),
        \vx_2
        \right>
        \\
        \vdots
        \\
        \vh_K \!=\!
        \displaystyle \sum_{
        \mathclap{
        \sigma \in \partitioning(K)
        }
        }
        \nu(\sigma) \left<
        \partial^{|\sigma|} \vh(\vx_0),
        \tensorprod{s \in \sigma} \vx_s
        \right>\!\!\!
      \end{pmatrix*}
    \\
    &\overset{\text{(\ref{eq:faa-di-bruno})}}{\to}
      \left(\!\!\!
      \begin{array}{l}
        \vg_0 \!=\!  \vg(\vh_0)
        \\
        \vg_1 \!=\! \left<
        \partial \vg(\vh_0),
        \vh_1
        \right>
        \\
        \vg_2 \!=\! \left<
        \partial^2 \vg(\vh_0),
        \vh_1^{\otimes 2}\right>
        \!+\!
        \left< \partial \vg(\vh_0),
        \vh_2
        \right>
        \\
        \vdots
        \\
        \vg_K \!=\!
        \displaystyle\sum_{
        \mathclap{
        \sigma \in \partitioning(K)
        }
        }
        \nu(\sigma) \left<
        \partial^{|\sigma|} \vg(\vh_0),
        \tensorprod{s \in \sigma} \vh_s
        \right>
      \end{array}
      \!\!\!\!
      \right)
      \overset{\eqref{eq:taylor-mode-scalar}}{=}
      \left(\!\!\!
      \begin{array}{l}
        \vf_0 \!=\!  \vf(\vx_0)
        \\
        \vf_1 \!=\! \left<
        \partial \vf(\vx_0),
        \vx_1
        \right>
        \\
        \vf_2 \!=\! \left<
        \partial^2 \vf(\vx_0),
        \vx_1^{\otimes 2}
        \right>
        \!+\!
        \left< \partial \vf(\vx_0),
        \vx_2
        \right>
        \\
        \vdots
        \\
        \vf_K \!=\!
        \displaystyle\sum_{
        \mathclap{
        \sigma \in \partitioning(K)
        }
        }
        \nu(\sigma) \left<
        \partial^{|\sigma|} \vf(\vx_0),
        \tensorprod{s \in \sigma} \vx_s
        \right>
      \end{array}
      \!\!\!\!
      \right)
  \end{split}
\end{align}
which describes the forward propagation of a \emph{single} $K$-jet.
However, computing popular PDE operators requires propagating \emph{multiple} $K$-jets in parallel, then summing their results.
We propose to pull this accumulation inside Taylor mode's propagation scheme, thereby collapsing it.

\begin{figure}[!t]
  \centering
  \begin{minipage}[t]{0.63\linewidth}
    \centering
    \vspace{0pt}
   \begin{tikzpicture}
    \tikzset{box/.style={rectangle, rounded corners, draw=black, inner sep=3pt, very thick}}
    \node[align=center, box] (topleft) {Extent input to \\ smooth path
      $\vx(t)$};

    \node[align=center, right=2.7cm of topleft, box] (topright) {Path in output \\ space $\vf(\vx(t))$};
    \draw [-Latex] (topleft.east) to node [midway, above] {$\vf$} (topright.west);

    \node[align=center, below=0.9cm of topright, box, draw=tab-orange, fill=tab-orange!25!white] (bottomright) {%Taylor polynomial of degree $K$
      % \\
      $K$-jet \; $\sum_{k=0}^K \frac{t^k}{k!} \vf_k$
      \\
      as $(\vf_0, \dots, \vf_K)$
    };
    \draw [-Latex] (topright.south) to node [midway, right] {$J^K$}
    (bottomright.north);

    \node[align=center, below=0.9cm of topleft, box, draw=tab-orange, fill=tab-orange!25!white] (bottomleft) {%Taylor polynomial of degree $K$
        % \\
        $K$-jet \; $\sum_{k=0}^K \frac{t^k}{k!} \vx_k$
        \\
        as $(\vx_0, \dots, \vx_K)$
      };
    \draw [-Latex] (topleft.south) to node [midway, left] {$J^K$} (bottomleft.north);

    \draw [-Latex, \colorTM, align=center, very thick] (bottomleft.east) to node [midway, above, tab-orange] {\color{\colorTM}Taylor mode} (bottomright.west);
  \end{tikzpicture}
  \end{minipage}
  \hfill
  \begin{minipage}[t]{0.35\linewidth}
    \caption{\textbf{Taylor mode propagates Taylor coefficients of a path in input space.}
    This results in the function-transformed path's Taylor coefficients.
    The Taylor expansion of degree $K$ is called a $K$-jet; hence Taylor mode propagates the input $K$-jet to the output $K$-jet.}
  \label{fig:utp}
  \end{minipage}
\end{figure}

%%% Local Variables:
%%% mode: LaTeX
%%% TeX-master: "../main"
%%% End:


%%% Local Variables:
%%% mode: LaTeX
%%% TeX-master: "../main"
%%% End:
