\begin{table}[!t]
  \centering
  \caption{\textbf{Benchmark from \cref{fig:benchmark} in numbers.}
    We fit linear functions and report their slopes, \ie, how much runtime and memory increase when incrementing the batch size or random samples.
    We show two significant digits and bold values are best according to parenthesized values.}
  \label{tab:benchmark}
  \vspace{1.5ex}
  % paths where the performances are stored
  \def\datapathLaplacianExact{jet/exp/exp01_benchmark_laplacian/performance/architecture_tanh_mlp_768_768_512_512_1_device_cuda_dim_50_name_laplacian_vary_batch_size}
  \def\datapathLaplacianStochastic{jet/exp/exp01_benchmark_laplacian/performance/architecture_tanh_mlp_768_768_512_512_1_batch_size_2048_device_cuda_dim_50_distribution_normal_name_laplacian_vary_num_samples}
  \def\datapathWeightedLaplacianExact{jet/exp/exp01_benchmark_laplacian/performance/architecture_tanh_mlp_768_768_512_512_1_device_cuda_dim_50_name_weighted_laplacian_vary_batch_size_rank_ratio_1_0}
  \def\datapathWeightedLaplacianStochastic{jet/exp/exp01_benchmark_laplacian/performance/architecture_tanh_mlp_768_768_512_512_1_batch_size_2048_device_cuda_dim_50_distribution_normal_name_weighted_laplacian_vary_num_samples_rank_ratio_1_0}
  \def\datapathBilaplacianExact{jet/exp/exp01_benchmark_laplacian/performance/architecture_tanh_mlp_768_768_512_512_1_device_cuda_dim_5_name_bilaplacian_vary_batch_size}
  \def\datapathBilaplacianStochastic{jet/exp/exp01_benchmark_laplacian/performance/architecture_tanh_mlp_768_768_512_512_1_batch_size_256_device_cuda_dim_5_distribution_normal_name_bilaplacian_vary_num_samples}
  % configuration options for the \num command
  \sisetup{%
    % scientific-notation=true,%
    round-mode=figures,%
    round-precision=2,%
    detect-weight, % for bolding to work
    tight-spacing=true, % less space around \cdot
  }
  \begin{small}
    \begin{tabular}{ccc|cccc}
      \toprule
      \textbf{Mode}
      & \makecell{\textbf{Per-datum or } \\ \textbf{-sample cost}}
      & \textbf{Implementation}
      & \textbf{Laplacian}
      & \makecell{\textbf{Weighted} \\ \textbf{Laplacian}}
      & \textbf{Biharmonic}
      \\
      \midrule
      \multirow{9}{*}{\textbf{Exact}}
      & \multirow{3}{*}{Time [ms]}
      & \textcolor{tab-blue}{Nested 1\textsuperscript{st}-order}
      & \input{\datapathLaplacianExact/hessian_trace_best.txt}
      & \input{\datapathWeightedLaplacianExact/hessian_trace_best.txt}
      & \input{\datapathBilaplacianExact/hessian_trace_best.txt}
      \\
      &
      & \textcolor{tab-orange}{Standard Taylor}
      & \input{\datapathLaplacianExact/jet_naive_best.txt}
      & \input{\datapathWeightedLaplacianExact/jet_naive_best.txt}
      & \input{\datapathBilaplacianExact/jet_naive_best.txt}
      \\
      &
      & \textcolor{tab-green}{Collapsed (ours)}
      & \textbf{\input{\datapathLaplacianExact/jet_simplified_best.txt}}
      & \textbf{\input{\datapathWeightedLaplacianExact/jet_simplified_best.txt}}
      & \textbf{\input{\datapathBilaplacianExact/jet_simplified_best.txt}}
      \\ \cmidrule{2-6}
      & \multirow{3}{*}{\makecell{Mem.\,[MiB] \\ (differentiable)}}
      & \textcolor{tab-blue}{Nested 1\textsuperscript{st}-order}
      & \input{\datapathLaplacianExact/hessian_trace_peakmem.txt}
      & \input{\datapathWeightedLaplacianExact/hessian_trace_peakmem.txt}
      & \input{\datapathBilaplacianExact/hessian_trace_peakmem.txt}
      \\
      &
      & \textcolor{tab-orange}{Standard Taylor}
      & \input{\datapathLaplacianExact/jet_naive_peakmem.txt}
      & \input{\datapathWeightedLaplacianExact/jet_naive_peakmem.txt}
      & \input{\datapathBilaplacianExact/jet_naive_peakmem.txt}
      \\
      &
      & \textcolor{tab-green}{Collapsed (ours)}
      & \textbf{\input{\datapathLaplacianExact/jet_simplified_peakmem.txt}}
      & \textbf{\input{\datapathWeightedLaplacianExact/jet_simplified_peakmem.txt}}
      & \textbf{\input{\datapathBilaplacianExact/jet_simplified_peakmem.txt}}
      \\ \cmidrule{2-6}
      & \multirow{3}{*}{\makecell{Mem.\,[MiB] \\ (non-diff.)}}
      & \textcolor{tab-blue}{Nested 1\textsuperscript{st}-order}
      & \input{\datapathLaplacianExact/hessian_trace_peakmem_nondifferentiable.txt}
      & \input{\datapathWeightedLaplacianExact/hessian_trace_peakmem_nondifferentiable.txt}
      & \input{\datapathBilaplacianExact/hessian_trace_peakmem_nondifferentiable.txt}
      \\
      &
      & \textcolor{tab-orange}{Standard Taylor}
      & \input{\datapathLaplacianExact/jet_naive_peakmem_nondifferentiable.txt}
      & \input{\datapathWeightedLaplacianExact/jet_naive_peakmem_nondifferentiable.txt}
      & \input{\datapathBilaplacianExact/jet_naive_peakmem_nondifferentiable.txt}
      \\
      &
      & \textcolor{tab-green}{Collapsed (ours)}
      & \textbf{\input{\datapathLaplacianExact/jet_simplified_peakmem_nondifferentiable.txt}}
      & \textbf{\input{\datapathWeightedLaplacianExact/jet_simplified_peakmem_nondifferentiable.txt}}
      & \textbf{\input{\datapathBilaplacianExact/jet_simplified_peakmem_nondifferentiable.txt}}
      \\
      \midrule
      \multirow{9}{*}{\textbf{Stochastic}}
      & \multirow{3}{*}{Time [ms]}
      & \textcolor{tab-blue}{Nested 1\textsuperscript{st}-order}
      & \input{\datapathLaplacianStochastic/hessian_trace_best.txt}
      & \input{\datapathWeightedLaplacianStochastic/hessian_trace_best.txt}
      & \input{\datapathBilaplacianStochastic/hessian_trace_best.txt}
      \\
      &
      & \textcolor{tab-orange}{Standard Taylor}
      & \input{\datapathLaplacianStochastic/jet_naive_best.txt}
      & \input{\datapathWeightedLaplacianStochastic/jet_naive_best.txt}
      & \input{\datapathBilaplacianStochastic/jet_naive_best.txt}
      \\
      &
      & \textcolor{tab-green}{Collapsed (ours)}
      & \textbf{\input{\datapathLaplacianStochastic/jet_simplified_best.txt}}
      & \textbf{\input{\datapathWeightedLaplacianStochastic/jet_simplified_best.txt}}
      & \textbf{\input{\datapathBilaplacianStochastic/jet_simplified_best.txt}}
      \\ \cmidrule{2-6}
      & \multirow{3}{*}{\makecell{Mem.\,[MiB]\\(differentiable)}}
      & \textcolor{tab-blue}{Nested 1\textsuperscript{st}-order}
      & \input{\datapathLaplacianStochastic/hessian_trace_peakmem.txt}
      & \input{\datapathWeightedLaplacianStochastic/hessian_trace_peakmem.txt}
      & \input{\datapathBilaplacianStochastic/hessian_trace_peakmem.txt}
      \\
      &
      & \textcolor{tab-orange}{Standard Taylor}
      & \input{\datapathLaplacianStochastic/jet_naive_peakmem.txt}
      & \input{\datapathWeightedLaplacianStochastic/jet_naive_peakmem.txt}
      & \input{\datapathBilaplacianStochastic/jet_naive_peakmem.txt}
      \\
      &
      & \textcolor{tab-green}{Collapsed (ours)}
      & \textbf{\input{\datapathLaplacianStochastic/jet_simplified_peakmem.txt}}
      & \textbf{\input{\datapathWeightedLaplacianStochastic/jet_simplified_peakmem.txt}}
      & \textbf{\input{\datapathBilaplacianStochastic/jet_simplified_peakmem.txt}}
      \\ \cmidrule{2-6}
      & \multirow{3}{*}{\makecell{Mem.\,[MiB]\\(non-diff.)}}
      & \textcolor{tab-blue}{Nested 1\textsuperscript{st}-order}
      & \input{\datapathLaplacianStochastic/hessian_trace_peakmem_nondifferentiable.txt}
      & \input{\datapathWeightedLaplacianStochastic/hessian_trace_peakmem_nondifferentiable.txt}
      & \input{\datapathBilaplacianStochastic/hessian_trace_peakmem_nondifferentiable.txt}
      \\
      &
      & \textcolor{tab-orange}{Standard Taylor}
      & \input{\datapathLaplacianStochastic/jet_naive_peakmem_nondifferentiable.txt}
      & \input{\datapathWeightedLaplacianStochastic/jet_naive_peakmem_nondifferentiable.txt}
      & \input{\datapathBilaplacianStochastic/jet_naive_peakmem_nondifferentiable.txt}
      \\
      &
      & \textcolor{tab-green}{Collapsed (ours)}
      & \textbf{\input{\datapathLaplacianStochastic/jet_simplified_peakmem_nondifferentiable.txt}}
      & \textbf{\input{\datapathWeightedLaplacianStochastic/jet_simplified_peakmem_nondifferentiable.txt}}
      & \textbf{\input{\datapathBilaplacianStochastic/jet_simplified_peakmem_nondifferentiable.txt}}
      \\
      \bottomrule
    \end{tabular}
  \end{small}
\end{table}
%%% Local Variables:
%%% mode: LaTeX
%%% TeX-master: "../main"
%%% End:


Here, we describe our implementation of the Taylor mode collapsing process and empirically validate its performance improvements on the previously discussed operators.

\paragraph{Design decisions \& limitations.}
JAX~\cite{bradbury2018jax} already offers an---albeit experimental---Taylor mode implementation~\cite{bettencourt2019taylor}.
However, we found it challenging to capture the computation graph and modify it using JAX's public interface.
In contrast, PyTorch \cite{paszke2019pytorch} provides \texttt{torch.fx} \cite{reed2022torch}, which offers a user-friendly interface to capture and transform computational graphs purely in Python.
Hence, we re-implemented Taylor mode in PyTorch, taking heavy inspiration from the JAX implementation.

This deliberate choice imposes certain limitations.
First, as of now, our Taylor mode in PyTorch supports only a small number of primitives, because the Taylor arithmetic in \cref{eq:faa-di-bruno} needs to be implemented case by case (this of course also applies to JAX's Taylor mode, which has broader operator coverage).
Second, while our Taylor mode implementation is competitive with JAX's, we did not fully optimize it (\eg, we do \emph{not} use in-place operations, and we do \emph{not} implement the efficient schemes from \citet[][\S13]{griewank2008evaluating}, but stick to Fa\`a di Bruno (\cref{eq:faa-di-bruno})).
Given our implementation's superiority compared to nested first-order AD that we demonstrate below, these are promising future efforts that will further improve performance, and we believe that making Taylor mode available to the PyTorch community is also an important step towards establishing its use.

\paragraph{Usage (overview in \cref{sec:appendix-visual-tour}).}
Our implementation takes a PyTorch function (\eg, a neural net) and first captures its computational graph using \texttt{torch.fx}'s symbolic tracing mechanism.
Then, it replaces each operation with its Taylor arithmetic, which yields the computational graph of the function's $K$-jet.
Users can then write a function to compute their differential operator with this vanilla Taylor mode.
Collapsing is achieved using a function \texttt{simplify}, which traces the computation again, rewrites the graph, and propagates the summation of highest coefficients up to its leafs.
This requires one backward traversal through the graph (\cref{sec:graph-simplifications} presents a detailed example).
The simplified graph produces the same result, but propagates summed coefficients, \ie, uses collapsed Taylor mode.

\paragraph{Experimental setup.}
We empirically validate our proposed collapsing approach in PyTorch.
We compare \textcolor{tab-orange}{standard Taylor mode} with \textcolor{tab-green}{collapsed Taylor mode} and \textcolor{tab-blue}{nested 1\textsuperscript{st}-order AD} on an Nvidia RTX 6000 GPU with 24 GiB memory.
To implement the (weighted) Laplacian and its stochastic counterpart, we use vector-Hessian-vector products (VHVPs) in forward-over-reverse order, as recommended \cite{griewank2008evaluating,dagreou2024how}.
For the biharmonic operator, we simply nest two VHVPs.
For the weighted Laplacian's coefficient matrix, we choose a full-rank diagonal matrix (\cref{sec:rank-deficient-weighted-laplacian} shows results for rank-deficient weightings).
To avoid confounding factors, all implementations are executed without compilation (our JAX experiments with the Laplacian in \cref{sec:jax-benchmark} confirm that \texttt{jit} does not affect the relative performance).
As common for PINNs \cite[\eg,][]{shi2024stochastic,dangel2024kroneckerfactored}, we use a 5-layer MLP $\smash{f_\vtheta}: D \to 768 \to 768 \to 512 \to 512 \to 1$ with $\tanh$ activations and trainable parameters $\vtheta$, and compute the PDE operators on batches of size $N$.
We measure three performance metrics:
\textbf{(1) runtime} reports the smallest execution time of 50 repetitions.
\textbf{(2) Peak memory (non-differentiable)} measures the maximum allocated GPU memory when computing the PDE operator's value (\eg, used in VMC \cite{pfau2020ab}) inside a \texttt{torch.no\_grad} context.
\textbf{(3) Peak memory (differentiable)} is the maximum memory usage when computing the PDE operator inside a \texttt{torch.enable\_grad} context, which allows backpropagation to $\vtheta$ (required for training PINNs, or alternative VMC works \cite{webber2022rayleigh, toulouse2007optimization}). This demands saving intermediates, which uses more memory but does not affect runtime.
As memory allocation does not fluctuate much, we measure it in a single run.

\begin{table}[!t]
  \centering
  \caption{\textbf{Benchmark from \cref{fig:benchmark} in numbers.}
    We fit linear functions and report their slopes, \ie, how much runtime and memory increase when incrementing the batch size or random samples.
    We show two significant digits and bold values are best according to parenthesized values.}
  \label{tab:benchmark}
  \vspace{1.5ex}
  % paths where the performances are stored
  \def\datapathLaplacianExact{jet/exp/exp01_benchmark_laplacian/performance/architecture_tanh_mlp_768_768_512_512_1_device_cuda_dim_50_name_laplacian_vary_batch_size}
  \def\datapathLaplacianStochastic{jet/exp/exp01_benchmark_laplacian/performance/architecture_tanh_mlp_768_768_512_512_1_batch_size_2048_device_cuda_dim_50_distribution_normal_name_laplacian_vary_num_samples}
  \def\datapathWeightedLaplacianExact{jet/exp/exp01_benchmark_laplacian/performance/architecture_tanh_mlp_768_768_512_512_1_device_cuda_dim_50_name_weighted_laplacian_vary_batch_size_rank_ratio_1_0}
  \def\datapathWeightedLaplacianStochastic{jet/exp/exp01_benchmark_laplacian/performance/architecture_tanh_mlp_768_768_512_512_1_batch_size_2048_device_cuda_dim_50_distribution_normal_name_weighted_laplacian_vary_num_samples_rank_ratio_1_0}
  \def\datapathBilaplacianExact{jet/exp/exp01_benchmark_laplacian/performance/architecture_tanh_mlp_768_768_512_512_1_device_cuda_dim_5_name_bilaplacian_vary_batch_size}
  \def\datapathBilaplacianStochastic{jet/exp/exp01_benchmark_laplacian/performance/architecture_tanh_mlp_768_768_512_512_1_batch_size_256_device_cuda_dim_5_distribution_normal_name_bilaplacian_vary_num_samples}
  % configuration options for the \num command
  \sisetup{%
    % scientific-notation=true,%
    round-mode=figures,%
    round-precision=2,%
    detect-weight, % for bolding to work
    tight-spacing=true, % less space around \cdot
  }
  \begin{small}
    \begin{tabular}{ccc|cccc}
      \toprule
      \textbf{Mode}
      & \makecell{\textbf{Per-datum or } \\ \textbf{-sample cost}}
      & \textbf{Implementation}
      & \textbf{Laplacian}
      & \makecell{\textbf{Weighted} \\ \textbf{Laplacian}}
      & \textbf{Biharmonic}
      \\
      \midrule
      \multirow{9}{*}{\textbf{Exact}}
      & \multirow{3}{*}{Time [ms]}
      & \textcolor{tab-blue}{Nested 1\textsuperscript{st}-order}
      & \input{\datapathLaplacianExact/hessian_trace_best.txt}
      & \input{\datapathWeightedLaplacianExact/hessian_trace_best.txt}
      & \input{\datapathBilaplacianExact/hessian_trace_best.txt}
      \\
      &
      & \textcolor{tab-orange}{Standard Taylor}
      & \input{\datapathLaplacianExact/jet_naive_best.txt}
      & \input{\datapathWeightedLaplacianExact/jet_naive_best.txt}
      & \input{\datapathBilaplacianExact/jet_naive_best.txt}
      \\
      &
      & \textcolor{tab-green}{Collapsed (ours)}
      & \textbf{\input{\datapathLaplacianExact/jet_simplified_best.txt}}
      & \textbf{\input{\datapathWeightedLaplacianExact/jet_simplified_best.txt}}
      & \textbf{\input{\datapathBilaplacianExact/jet_simplified_best.txt}}
      \\ \cmidrule{2-6}
      & \multirow{3}{*}{\makecell{Mem.\,[MiB] \\ (differentiable)}}
      & \textcolor{tab-blue}{Nested 1\textsuperscript{st}-order}
      & \input{\datapathLaplacianExact/hessian_trace_peakmem.txt}
      & \input{\datapathWeightedLaplacianExact/hessian_trace_peakmem.txt}
      & \input{\datapathBilaplacianExact/hessian_trace_peakmem.txt}
      \\
      &
      & \textcolor{tab-orange}{Standard Taylor}
      & \input{\datapathLaplacianExact/jet_naive_peakmem.txt}
      & \input{\datapathWeightedLaplacianExact/jet_naive_peakmem.txt}
      & \input{\datapathBilaplacianExact/jet_naive_peakmem.txt}
      \\
      &
      & \textcolor{tab-green}{Collapsed (ours)}
      & \textbf{\input{\datapathLaplacianExact/jet_simplified_peakmem.txt}}
      & \textbf{\input{\datapathWeightedLaplacianExact/jet_simplified_peakmem.txt}}
      & \textbf{\input{\datapathBilaplacianExact/jet_simplified_peakmem.txt}}
      \\ \cmidrule{2-6}
      & \multirow{3}{*}{\makecell{Mem.\,[MiB] \\ (non-diff.)}}
      & \textcolor{tab-blue}{Nested 1\textsuperscript{st}-order}
      & \input{\datapathLaplacianExact/hessian_trace_peakmem_nondifferentiable.txt}
      & \input{\datapathWeightedLaplacianExact/hessian_trace_peakmem_nondifferentiable.txt}
      & \input{\datapathBilaplacianExact/hessian_trace_peakmem_nondifferentiable.txt}
      \\
      &
      & \textcolor{tab-orange}{Standard Taylor}
      & \input{\datapathLaplacianExact/jet_naive_peakmem_nondifferentiable.txt}
      & \input{\datapathWeightedLaplacianExact/jet_naive_peakmem_nondifferentiable.txt}
      & \input{\datapathBilaplacianExact/jet_naive_peakmem_nondifferentiable.txt}
      \\
      &
      & \textcolor{tab-green}{Collapsed (ours)}
      & \textbf{\input{\datapathLaplacianExact/jet_simplified_peakmem_nondifferentiable.txt}}
      & \textbf{\input{\datapathWeightedLaplacianExact/jet_simplified_peakmem_nondifferentiable.txt}}
      & \textbf{\input{\datapathBilaplacianExact/jet_simplified_peakmem_nondifferentiable.txt}}
      \\
      \midrule
      \multirow{9}{*}{\textbf{Stochastic}}
      & \multirow{3}{*}{Time [ms]}
      & \textcolor{tab-blue}{Nested 1\textsuperscript{st}-order}
      & \input{\datapathLaplacianStochastic/hessian_trace_best.txt}
      & \input{\datapathWeightedLaplacianStochastic/hessian_trace_best.txt}
      & \input{\datapathBilaplacianStochastic/hessian_trace_best.txt}
      \\
      &
      & \textcolor{tab-orange}{Standard Taylor}
      & \input{\datapathLaplacianStochastic/jet_naive_best.txt}
      & \input{\datapathWeightedLaplacianStochastic/jet_naive_best.txt}
      & \input{\datapathBilaplacianStochastic/jet_naive_best.txt}
      \\
      &
      & \textcolor{tab-green}{Collapsed (ours)}
      & \textbf{\input{\datapathLaplacianStochastic/jet_simplified_best.txt}}
      & \textbf{\input{\datapathWeightedLaplacianStochastic/jet_simplified_best.txt}}
      & \textbf{\input{\datapathBilaplacianStochastic/jet_simplified_best.txt}}
      \\ \cmidrule{2-6}
      & \multirow{3}{*}{\makecell{Mem.\,[MiB]\\(differentiable)}}
      & \textcolor{tab-blue}{Nested 1\textsuperscript{st}-order}
      & \input{\datapathLaplacianStochastic/hessian_trace_peakmem.txt}
      & \input{\datapathWeightedLaplacianStochastic/hessian_trace_peakmem.txt}
      & \input{\datapathBilaplacianStochastic/hessian_trace_peakmem.txt}
      \\
      &
      & \textcolor{tab-orange}{Standard Taylor}
      & \input{\datapathLaplacianStochastic/jet_naive_peakmem.txt}
      & \input{\datapathWeightedLaplacianStochastic/jet_naive_peakmem.txt}
      & \input{\datapathBilaplacianStochastic/jet_naive_peakmem.txt}
      \\
      &
      & \textcolor{tab-green}{Collapsed (ours)}
      & \textbf{\input{\datapathLaplacianStochastic/jet_simplified_peakmem.txt}}
      & \textbf{\input{\datapathWeightedLaplacianStochastic/jet_simplified_peakmem.txt}}
      & \textbf{\input{\datapathBilaplacianStochastic/jet_simplified_peakmem.txt}}
      \\ \cmidrule{2-6}
      & \multirow{3}{*}{\makecell{Mem.\,[MiB]\\(non-diff.)}}
      & \textcolor{tab-blue}{Nested 1\textsuperscript{st}-order}
      & \input{\datapathLaplacianStochastic/hessian_trace_peakmem_nondifferentiable.txt}
      & \input{\datapathWeightedLaplacianStochastic/hessian_trace_peakmem_nondifferentiable.txt}
      & \input{\datapathBilaplacianStochastic/hessian_trace_peakmem_nondifferentiable.txt}
      \\
      &
      & \textcolor{tab-orange}{Standard Taylor}
      & \input{\datapathLaplacianStochastic/jet_naive_peakmem_nondifferentiable.txt}
      & \input{\datapathWeightedLaplacianStochastic/jet_naive_peakmem_nondifferentiable.txt}
      & \input{\datapathBilaplacianStochastic/jet_naive_peakmem_nondifferentiable.txt}
      \\
      &
      & \textcolor{tab-green}{Collapsed (ours)}
      & \textbf{\input{\datapathLaplacianStochastic/jet_simplified_peakmem_nondifferentiable.txt}}
      & \textbf{\input{\datapathWeightedLaplacianStochastic/jet_simplified_peakmem_nondifferentiable.txt}}
      & \textbf{\input{\datapathBilaplacianStochastic/jet_simplified_peakmem_nondifferentiable.txt}}
      \\
      \bottomrule
    \end{tabular}
  \end{small}
\end{table}
%%% Local Variables:
%%% mode: LaTeX
%%% TeX-master: "../main"
%%% End:


\paragraph{Results.}
\Cref{fig:benchmark} visualizes the growth in computational resources \wrt the batch size (exact) and random samples (stochastic) for fixed dimensions $D$.
Runtime and memory increase linearly in both, as expected.
We quantify the results by fitting linear functions and reporting their slopes (\ie, time and memory added per datum/sample) in \cref{tab:benchmark}.
We make the following observations:
\begin{itemize}[leftmargin=0.5cm]
\item \textbf{Collapsed Taylor mode accelerates standard Taylor mode.}
  The measured performance differences correspond well with the theoretical estimate from counting the number of forward-propagated vectors.
  \Eg, for the exact Laplacian, adding one datum introduces {\color{tab-green}$2 + D$} versus {\color{tab-orange}$1 + 2D$} new vectors.
  For $D=50$, their ratio is $\nicefrac{\color{tab-green}(2 + D)}{\color{tab-orange}(1 + 2D)} \approx 0.51$.
Empirically, we measure that adding one datum adds {\color{tab-orange}\inputMetricOnly{jet/exp/exp01_benchmark_laplacian/performance/architecture_tanh_mlp_768_768_512_512_1_device_cuda_dim_50_name_laplacian_vary_batch_size/jet_naive_best.txt}\,ms} to standard, and {\color{tab-green}\inputMetricOnly{jet/exp/exp01_benchmark_laplacian/performance/architecture_tanh_mlp_768_768_512_512_1_device_cuda_dim_50_name_laplacian_vary_batch_size/jet_simplified_best.txt}\,ms} to collapsed, Taylor mode (\cref{tab:benchmark}); the ratio of $\approx \!\!\!\inputMetricRatio{jet/exp/exp01_benchmark_laplacian/performance/architecture_tanh_mlp_768_768_512_512_1_device_cuda_dim_50_name_laplacian_vary_batch_size/jet_simplified_best.txt}{jet/exp/exp01_benchmark_laplacian/performance/architecture_tanh_mlp_768_768_512_512_1_device_cuda_dim_50_name_laplacian_vary_batch_size/jet_naive_best.txt}$ is close.
  Similar arguments hold for peak memory of differentiable computation, stochastic approximation, and the other PDE operators (see \cref{tab:benchmark-ratios} for all comparisons).

\item \textbf{Collapsed Taylor mode outperforms nested 1\textsuperscript{st}-order AD.}
  For the exact and stochastic (weighted) Laplacians, collapsed Taylor mode is roughly twice as fast (consistent with the JAX results in \cref{fig:vanilla-taylor-not-enough}) while using only 40-50\% memory.
  For the biharmonic operator, we also observe speed-ups;
  in the stochastic case up to 9x in time, and 5x in memory (differentiable).
\end{itemize}

%Because Taylor mode performs a single forward propagation, in contrast to nested differentiation, which needs to store intermediates for future differentiation, its memory consumption is much better, specifically for non-differentiable computations.

\paragraph{Comparison with JAX.} We also conducted experiments with JAX (+ \texttt{jit}) to rule out artifacts from choosing PyTorch, implementation mistakes in our Taylor mode library, or unexpected simplifications from the JIT compiler.
We find that the choice of the ML framework does not affect the results.
\Eg, when computing the exact Laplacian with nested first-order AD, PyTorch consumes \inputMetricOnly{jet/exp/exp01_benchmark_laplacian/performance/architecture_tanh_mlp_768_768_512_512_1_device_cuda_dim_50_name_laplacian_vary_batch_size/hessian_trace_best.txt}\,ms per datum (\cref{tab:benchmark}), while JAX uses \inputMetricOnly{jet/exp/exp04_jax_benchmark/performance/architecture_tanh_mlp_768_768_512_512_1_device_cuda_dim_50_name_jax_laplacian_vary_batch_size/hessian_trace_best.txt}\,ms (\cref{fig:vanilla-taylor-not-enough,tab:jax-benchmark}).
We find the same trend when comparing our collapsed Taylor mode and JAX's forward Laplacian.
Interestingly, we noticed that JAX's Taylor mode was consistently slower than our PyTorch implementation, despite using \texttt{jit}.
We hypothesize that this could stem from algorithmic differences in the Taylor mode implementations and conclude from these results that (both ours, as well as the existing JAX) Taylor mode still has potential for improvements that may further increase the margin to nested first-order.

%%% Local Variables:
%%% mode: LaTeX
%%% TeX-master: "../main"
%%% End:
